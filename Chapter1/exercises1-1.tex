\documentclass{article}

\renewcommand{\thesection}{}	% Remove section labels

\usepackage{amsmath}

\begin{document}

	\section{1.1 Exercises}
	
	\begin{enumerate}
		\item If $f(x) = x  + \sqrt{2 - x}$ and $g(u) = u + \sqrt{2 - u}$ is it true that $f = g$?
		
			Yes, in this case the functions are equivalent.
			
		\item If $f(x) = \frac{x^2 - x}{x - 1}$ and $g(x) = x$, is it true that $f = g$.
		
			No. They are almost the same however as $x = 1$ is undefined for $f(x) = \frac{x^2 - x}{x - 1}$ 
			then $f != g$.
			
		\item The graph of a function $f$ is given.
		\begin{enumerate}
			\item State the value of $f(-1)$.
				
				$$f(-1) = -2$$
				
			\item Estimate the value of $f(2)$.
			
				2.8.
				
			\item For what values of $x$ is $f(x) = 2$?
			
				-3 and 1.
				
			\item Estimate the value of \emph{x} such that $f(x) = 0$.
			
				-2.5 and 0.3
				
			\item State the domain and range of \emph{f}.
			
				Domain: $[-3 , 3]$, Range: $[-2, 3]$
				
			\item On what interval is $f$ increasing?
			
				$$(-1, 3)$$
		\end{enumerate}
		
		\item The graphs of \emph{f} and \emph{g} are given.
		
		\begin{enumerate}
			\item State the values of $f(-4)$ and $g(3)$.
			
				$$f(-4) = -2 \text{ and } g(3) = 4$$
				
			\item For what values of \emph{x} is $f(x) = g(x)$?
			
				-2 and 2.
				
			\item Estimate the solution of the equation $f(x) = -1$.
			
				$$x = -3 \text{ or } 4$$
			
			\item On what interval is $f$ decreasing?
			
				$$(0, 4]$$
				
			\item State the domain and range of \emph{f}.
			
				$$ \text{Doman: } [-4, 4] \text{ Range: } [-2, 2 ]$$
				
			\item State the domain and range of \emph{g}.
			
				$$ \text{Domain: } [-4, 3] \text{ Range: } [0.5, 4]$$
		\end{enumerate}
		
		\item[5-8] Determine whether the curve is the graph of a function of \emph{x}. If
			it is, state the domain and range of the function.
			
		\item This curve is not a function of \emph{x} as it fails the vertical line test.
		
		\item This curve is a function of \emph{x}.
		
			$$ \text{Domain: } [ -2, 2] \text{ Range: } [-1, 2]$$
			
		\item This curve is a function of \emph{x}.
		
			$$ \text{Domain: } [-3, 2] \text{ Range: } [-3, 2) \cup [-1, 3]$$
			
		\item This curve fails the vertical line test and so is not a function of \emph{x}.
		
		\item The graph shown gives the weight of a certain person as a function of age.
			Describe in words hiw this person's weight varies over time. What do you think
			happened when this person was 30 years old?
			
			From age 0 to 12 the weight increased at a steady rate from about 5 kg to about
			20 kg. Then the rate of increase became greater and from 12 to 20 the individual
			went from 20 kg to 60 kg. After this the rate of increase became much lower until 
			the age of 30, gaining about only 2 kg. At the there was a large decrease in weight
			to about 50 kg until the age of about 37 when the individual weight rose to about 63 kg.
			From the age of 40 to 70 there was a steady increase in weight to nearly 80 kg.
			
		\item The graph shows the height of the water in a bathtub as a functionof time. Give a verbal
			description of what you think happened.
			
			 From t = 0 to t = 5, the bath was being filled with warm water. At t = 5 until 6, the cold
			 tap was also turned on, increasing the rate of water level time. At t = 6, both taps were
			 turned off. At t = 13, a the plug was pulled and a little water was let out of the bath tub
			 and then immediately refilled from the tap. At t = 18 the plug was pulled and the bath 
			 was emptied.
			 
		\item You put some ice cubes in a glass, fill the glass with cold water, and then let the glass sit on 
			a table. Describe how the temperature of the water changes as time passes. Then sketch a
			rough graph of the temperature of the water as a function of elapsed time.
			
		\item Sketch a rough graph of the number of hours of daylight as a function of the time of year.
		
		\item Sketch a rough graph of the outdoor temperature as a function of time during a typical
			spring day.
			
		\item Sketch a rough graph of the market value of a new car as a function of time for a period of 
			20 years. Assume the car is well maintained.
			
		\item Sketch the graph of the amout of a particular brand of coffee sold by a store as a function
			of the price of the coffee.
			
		\item You place a frozen pie in an oven and bake it for an hour. Then you let it out and let it cool before
			eating it. Describe how the temperature of the pie changes as time passes. Then sketch a rough
			graph of the temperature as a function of time.
			
		\item  A homeowner mows the lawn every Wednesday afternoon. Sketch a rough graph of the height of
			the grass as a function of time over a four-week period.
			
		\item An airplane takes off from an airport and lands an hour later at another airport, 400 km away.
			If \emph{t} represents the time in minutes since the plane has left the terminal building, let
			$x(t)$ be the horizontal distance travelled and $y(t)$ be the altitude of the plane.
			
			\begin{enumerate}
				\item Sketch a possible graph of $x(t)$.
				
				\item Sketch a possible graph of $y(t)$.
				
				\item Sketch a possible graph of the ground speed.
				
				\item Sketch a possible graph of the vertical velocity.
			\end{enumerate}
			
		\item If $f(x) = 3x^2 - x + 2$, find 
			\begin{enumerate}
				\item $f(2)$
					$$3(2)^2 - (2) + 2$$
					$$3(4)$$
					$$f(2) = 12$$
					
				\item $f(-2)$
				
					$$3(-2)^2 - (-2) + 2$$
					$$3(4) + 2 + 2 $$
					$$12 + 4$$
					$$f(-2) = 16$$
					
				\item $f(a)$
				
					$$ f(a) = 3a^2 - a + 2 $$
				
				\item $f(-a)$
				
					$$3(-a)^2 - (- a) + 2$$
					$$f(-a) = 3a^2 + a + 2$$
					
				\item $f(a+ 1)$
				
					$$3(a + 1)^2 - (a + 1) + 2$$
					$$3(a^2 + 2a + 1) - a - 1 + 2 $$
					$$3a^2 + 6a + 3 - a + 1$$
					$$f(a+1) = 3a^2 +5a + 4$$
					
				\item $2f(a)$
				
					$$2 (3a^2 - a + 2) = 6a^2 - 2a + 4$$
					
				\item $f(2a)$
				
					$$3(2a)^2 - (2a) + 2$$
					$$3(4a^2) - 2a + 2$$
					$$12a^2 - 2a + 2$$
					
				\item $f(a^2)$
				
					$$3(a^2)^2 - (a^2) + 2$$
					$$3a^4 - a^2 + 2$$
			\end{enumerate}
	\end{enumerate}
\end{document}