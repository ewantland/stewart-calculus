\documentclass{article}

\renewcommand{\thesection}{}	% Remove section labels

\usepackage{amsmath}

\begin{document}

	\section{1.1 Exercises}
	
	\begin{enumerate}
		\item If $f(x) = x  + \sqrt{2 - x}$ and $g(u) = u + \sqrt{2 - u}$ is it true that $f = g$?
		
			Yes, in this case the functions are equivalent.
			
		\item If $f(x) = \frac{x^2 - x}{x - 1}$ and $g(x) = x$, is it true that $f = g$.
		
			No. They are almost the same however as $x = 1$ is undefined for $f(x) = \frac{x^2 - x}{x - 1}$ 
			then $f != g$.
			
		\item The graph of a function $f$ is given.
		\begin{enumerate}
			\item State the value of $f(-1)$.
				
				$$f(-1) = -2$$
				
			\item Estimate the value of $f(2)$.
			
				2.8.
				
			\item For what values of $x$ is $f(x) = 2$?
			
				-3 and 1.
				
			\item Estimate the value of \emph{x} such that $f(x) = 0$.
			
				-2.5 and 0.3
				
			\item State the domain and range of \emph{f}.
			
				Domain: $[-3 , 3]$, Range: $[-2, 3]$
				
			\item On what interval is $f$ increasing?
			
				$$(-1, 3)$$
		\end{enumerate}
		
		\item The graphs of \emph{f} and \emph{g} are given.
		
		\begin{enumerate}
			\item State the values of $f(-4)$ and $g(3)$.
			
				$$f(-4) = -2 \text{ and } g(3) = 4$$
				
			\item For what values of \emph{x} is $f(x) = g(x)$?
			
				-2 and 2.
				
			\item Estimate the solution of the equation $f(x) = -1$.
			
				$$x = -3 \text{ or } 4$$
			
			\item On what interval is $f$ decreasing?
			
				$$(0, 4]$$
				
			\item State the domain and range of \emph{f}.
			
				$$ \text{Doman: } [-4, 4] \text{ Range: } [-2, 2 ]$$
				
			\item State the domain and range of \emph{g}.
			
				$$ \text{Domain: } [-4, 3] \text{ Range: } [0.5, 4]$$
		\end{enumerate}
		
		\item[5-8] Determine whether the curve is the graph of a function of \emph{x}. If
			it is, state the domain and range of the function.
			
		\item This curve is not a function of \emph{x} as it fails the vertical line test.
		
		\item This curve is a function of \emph{x}.
		
			$$ \text{Domain: } [ -2, 2] \text{ Range: } [-1, 2]$$
			
		\item This curve is a function of \emph{x}.
		
			$$ \text{Domain: } [-3, 2] \text{ Range: } [-3, 2) \cup [-1, 3]$$
			
		\item This curve fails the vertical line test and so is not a function of \emph{x}.
		
		\item The graph shown gives the weight of a certain person as a function of age.
			Describe in words hiw this person's weight varies over time. What do you think
			happened when this person was 30 years old?
			
			From age 0 to 12 the weight increased at a steady rate from about 5 kg to about
			20 kg. Then the rate of increase became greater and from 12 to 20 the individual
			went from 20 kg to 60 kg. After this the rate of increase became much lower until 
			the age of 30, gaining about only 2 kg. At the there was a large decrease in weight
			to about 50 kg until the age of about 37 when the individual weight rose to about 63 kg.
			From the age of 40 to 70 there was a steady increase in weight to nearly 80 kg.
			
		\item The graph shows the height of the water in a bathtub as a functionof time. Give a verbal
			description of what you think happened.
			
			 From t = 0 to t = 5, the bath was being filled with warm water. At t = 5 until 6, the cold
			 tap was also turned on, increasing the rate of water level time. At t = 6, both taps were
			 turned off. At t = 13, a the plug was pulled and a little water was let out of the bath tub
			 and then immediately refilled from the tap. At t = 18 the plug was pulled and the bath 
			 was emptied.
			 
		\item You put some ice cubes in a glass, fill the glass with cold water, and then let the glass sit on 
			a table. Describe how the temperature of the water changes as time passes. Then sketch a
			rough graph of the temperature of the water as a function of elapsed time.
			
		\item Sketch a rough graph of the number of hours of daylight as a function of the time of year.
		
		\item Sketch a rough graph of the outdoor temperature as a function of time during a typical
			spring day.
			
		\item Sketch a rough graph of the market value of a new car as a function of time for a period of 
			20 years. Assume the car is well maintained.
			
		\item Sketch the graph of the amout of a particular brand of coffee sold by a store as a function
			of the price of the coffee.
			
		\item You place a frozen pie in an oven and bake it for an hour. Then you let it out and let it cool before
			eating it. Describe how the temperature of the pie changes as time passes. Then sketch a rough
			graph of the temperature as a function of time.
			
		\item  A homeowner mows the lawn every Wednesday afternoon. Sketch a rough graph of the height of
			the grass as a function of time over a four-week period.
			
		\item An airplane takes off from an airport and lands an hour later at another airport, 400 km away.
			If \emph{t} represents the time in minutes since the plane has left the terminal building, let
			$x(t)$ be the horizontal distance travelled and $y(t)$ be the altitude of the plane.
			
			\begin{enumerate}
				\item Sketch a possible graph of $x(t)$.
				
				\item Sketch a possible graph of $y(t)$.
				
				\item Sketch a possible graph of the ground speed.
				
				\item Sketch a possible graph of the vertical velocity.
			\end{enumerate}
			
		\item If $f(x) = 3x^2 - x + 2$, find 
			\begin{enumerate}
				\item $f(2)$
					$$3(2)^2 - (2) + 2$$
					$$3(4)$$
					$$f(2) = 12$$
					
				\item $f(-2)$
				
					$$3(-2)^2 - (-2) + 2$$
					$$3(4) + 2 + 2 $$
					$$12 + 4$$
					$$f(-2) = 16$$
					
				\item $f(a)$
				
					$$ f(a) = 3a^2 - a + 2 $$
				
				\item $f(-a)$
				
					$$3(-a)^2 - (- a) + 2$$
					$$f(-a) = 3a^2 + a + 2$$
					
				\item $f(a+ 1)$
				
					$$3(a + 1)^2 - (a + 1) + 2$$
					$$3(a^2 + 2a + 1) - a - 1 + 2 $$
					$$3a^2 + 6a + 3 - a + 1$$
					$$f(a+1) = 3a^2 +5a + 4$$
					
				\item $2f(a)$
				
					$$2 (3a^2 - a + 2) = 6a^2 - 2a + 4$$
					
				\item $f(2a)$
				
					$$3(2a)^2 - (2a) + 2$$
					$$3(4a^2) - 2a + 2$$
					$$12a^2 - 2a + 2$$
					
				\item $f(a^2)$
				
					$$3(a^2)^2 - (a^2) + 2$$
					$$3a^4 - a^2 + 2$$
					
				\item $[f(a)]^2$
				
					$$ (3x^2 - x + 2)^2$$
					$$9x^2 - 6x^3 + 13x^2 - 4x + 4$$
					
				\item $(f(a + h)$
					
					$$3(a + h)^2 - (a + h) + 2 $$
					$$3(a^2 + 2ah + h^2) - a - h + 2 $$
					$$3a^2 + 6ah + 3h^2 - a - h + 2 $$
			\end{enumerate}
			
			\item A spherical balloon with radius \emph{r} inches has volume $V(r) = \frac{4}{3}\pi  r^3$.
				Find a function that represents the amount of air required to inflate the balloon from a
				radius of \emph{r} inches to a radius $r + 1$ inches.
				
				We can use a difference quotient here.
				
				$$\frac{f(r + 1) - f(r)}{1}$$
				$$\frac{4}{3}\pi (r + 1)^3 - \frac{4}{3}\pi r^3$$
				$$\frac{4}{3}\pi (r^3 + 3r^2 + 3r + 1) - \frac{4}{3} \pi r^3$$
				$$\frac{4}{3} \pi (3r^2 + 3r + 1) $$
				
			\item[21-24] Evaluate the difference quotient for the given function. Simplify your answer.
			
		
			\item $f(x) = 4 + 3x - x^2$, $\frac{f(3 + h) - f(3)}{h}$

				$$\frac{[4 + 3(3 + h) - (3 + h)^2] - [4 + 3(3) - (3)^2]}{h}$$
				$$\frac{4 + 9 + 3h - h^2 - 6h - 9 - 4 - 9 + 9}{h}$$
				$$\frac{-h^2 - 3h}{h} = - h - 3$$

			\item $f(x) = x^3$, $\frac{f(a+h) - f(a)}{h}$

				$$\frac{a^3 + 3a^2h + 3ah^2 + h^3 - a^3}{h}$$
				$$\frac{3a^2h + 3ah^2 + h^3}{h}$$
				$$h^2 + 3ah + 3a^2$$

			\item $f(x) = \frac{1}{x}$, $\frac{f(x) - f(a)}{x - a}$

				$$\frac{\frac{1}{x}- \frac{1}{a}}{x-a}$$
				$$\frac{\frac{a-x}{ax}}{x - a}$$
				$$\frac{a - x}{ax(x-a)}$$
				$$-\frac{x-a}{ax(x - a)} = - \frac{1}{ax}$$

			\item $f(x) = \frac{x + 3}{x + 1}$, $\frac{f(x) - f(1)}{x - 1}$

				$$ \frac{ \frac{x + 3}{x + 1} - \frac{1 + 3}{1 + 1}}{x - 1} $$
				$$ \frac{ \frac{x + 3}{x +1} - 2}{x-1} $$
				$$ \frac{ \frac{x + 3 - 2(x + 1)}{x + 1}}{x - 1} $$
				$$ \frac{x + 3 - 2(x + 1)}{(x + 1)(x - 1)} $$
				$$ \frac{x + 1}{(x + 1)(x - 1)} = \frac{1}{x - 1} $$

			\item[25--29] Find the domain of the function.

			\item $f(x) = \frac{x + 4}{x^2 - 9} $

				$$ f(x) = \frac{x + 4}{x^2 - 9} = \frac{x + 4}{(x + 3)(x - 3)} $$
				$$ \text{Domain: } (-\infty, -3) \cup (-3, 3) \cup (3, \infty) $$

			\item $f(x) = \frac{2x^3 - 5}{x^2 + x - 6}$
				
				$$ \frac{2x^3 - 5}{x^2 + x - 6} = \frac{2x^3 - 5}{(x + 3)(x - 2)} $$
				$$ \text{Domain: } (-\infty, -3) \cup (-3, 2) \cup (2, \infty) $$

		\item $F(p) = \sqrt{2 - \sqrt{p}}$

			$$\text{We know that } \sqrt{p} \leq  2$$
			$$\text{and } p \geq 0$$
			$$\text{Domain: } [0, 4]$$

		\item $g(t) = \sqrt{3 - t} - \sqrt{2 + t}$

			$$\text{Because: } \sqrt{3 - t}, t \leq 3$$
			$$\text{Because: } \sqrt{2 + t}, t \geq -2$$
			$$\text{Domain: } [-2, 3]$$

		\item $h(x) = \frac{1}{\sqrt[4]{x^2 - 5x}}$

			$\sqrt[4]{x^2 - 5x}$ must not be zero. $x^2 - 5x$ must be positive.

			$$x^2 + 5x = x(x + 5)$$

			If we plot -ve and +ve values on a number line we find...

			$$\text{Domain: } (-\infty, 0) \cup (5, \infty) $$

		\item Find the domain and range and sketch the graph of the function $h(x) = \sqrt{4 - x^2}$.
		
		\item[31--42] Find the domain and sketch the graph of the function.

		\item $f(x) = 5$

		\item $F(x) = x^2 - 2x + 1$

		\item $f(t) = 2t + t^2$

		\item $H(t) = \frac{4 - t^2}{2 - t}$

		\item $g(x) = \sqrt{x - 5}$

		\item $F(x) = | 2x + 1 |$

		\item $G(x) = \frac{2x + |x|}{x}$

		\item $g(x) = \frac{ |x| }{x^2}$

		\item  

\begin{displaymath}
   f(x) = \left\{
     \begin{array}{lr}
       x + 2 & \text{if } x < 0\\
       1 - x &  \text{if } x \geq 0 
     \end{array}
   \right.
\end{displaymath} 

		\item

\begin{displaymath}
   f(x) = \left\{
     \begin{array}{lr}
       3 - \frac{1}{2}x & \text{if } x \leq 2\\
       2x - 5 &  \text{if } 2 > 0 
     \end{array}
   \right.
\end{displaymath} 

		\item

\begin{displaymath}
   f(x) = \left\{
     \begin{array}{lr}
       x + 2 & \text{if } x \leq -1\\
       x^2 &  \text{if } x > -1 
     \end{array}
   \right.
\end{displaymath} 

		\item

\begin{displaymath}
   f(x) = \left\{
     \begin{array}{lr}
       -1 & \text{if } x \leq -1\\
       3x + 2 &  \text{if } |x| < 1\\
       7 - 2x  & \text{if } x \geq 1
     \end{array}
   \right.
\end{displaymath} 

	\item[43--46] Find an expression for the function whose graph is
		the given curve.
		
	\item The line segment joining the points $(1, -3)$ and $(5, 7)$.
	
		First we need to find the gradient.
		
		$$m = \frac{-3 - 7}{1 - 5} = \frac{-10}{-4} = \frac{5}{2}$$
		
		$$(y - (-3)) = \frac{5}{2}(x - 1)$$
		
		$$y = \frac{5}{2}x - \frac{5}{2} - 3$$
		
		$$y = \frac{5}{2}x -\frac{11}{2}, 1 \leq x \leq 5$$
		
	\item The line segment joining the points $(-3, -2)$ and $(5, 7)$
	
		$$m = \frac{-2 - 7}{-3 - 5} = \frac{9}{8}$$
	
		$$(y - (-2)) = \frac{9}{8}(x - (-3))$$
		$$y + 2 = \frac{9}{8}x + \frac{27}{8}$$
		$$y = \frac{9}{8}x + \frac{11}{8}, -3 \leq x \leq 5$$
		
	\item The bottom half of the parabola $x + (y-1)^2 = 0$
		
		$$ (y - 1)^2 = -x$$
		$$y - 1 = \pm \sqrt{-z}$$
		
		We know we want the lower half of the parabola, so we use the negative square root.
		
		$$f(x) = 1 - \sqrt{-x}, x \leq 0$$
		
	\item The top half of the circle $x^2 + (y - 2)^2 = 4$
	
		$$ (y - 2)^2 = 4 - x^2$$
		$$ (y - 2) = \pm \sqrt{4 - x^2} $$
		
		We want the top half of the circle so we take the positive square root.
		
		$$ f(x) = 2 + \sqrt{4 - x^2} $$
		
		$$ 4 - x^2 \text{ must be } \geq 0$$
		
		$$4 - x^2 = (2 + x)(2 - x)$$
		
		$$\text{Domain: } [-2, 2]$$
		
	\item[47--51] Find a formula for the described function and state its domain
	
	\item A rectangle has perimeter 20 m. Express the area of the rectangle as a function of the
		length of one of its sides.
		
		$$2x + 2y = 20$$
		$$A(x) = xy$$
		
		Lets find \emph{y} in terms of \emph{x}.
		
		$$2y = 20 - 2x$$
		$$y = 10 - x$$
		
		Substitute into the area formula.
		
		$$A(x) = x(10 - x)$$
		$$A(x) = 10x - x^2, \text{ Domain: } (0, 10)$$
		
	\item A rectangle has area $16 \text{m}^2$. Express the perimeter of the rectangle as a
		function if the length of one of its side.
		
		$$xy = 16$$
		$$y = \frac{16}{x}$$
		
		$$P(x) = 2x + 2y$$
		$$P(x) = 2x + 2(\frac{16}{x})$$
		$$P(x) = 2x + \frac{32}{x}, \text{ Domain: } [0, \infty)$$
		
	\item Express the area of an equilateral triangle as a function of the length of a side.
	
		We know that all sides are equal and all angles are $60^{\circ}$.

		$$A = \frac{1}{2}ab\sin C$$
		$$ a = b, \text{ so } ab = x^2 $$
		$$\sin 60 = \frac{\sqrt{3}}{2}$$

		$$ A(x) = \frac{x^2\sqrt{3}}{4} \text{ Domain: } [0, \infty)$$

	\item Express the surface area of a cube as a function of its volume.

		$$V = x^3$$
		$$x = \sqrt[3]{V}$$
		$$A{x} = 6 x^2$$
		Substitue in V for x
		$$A(V) = 6 V^{\frac{2}{3}}$$
		$$\text{Domain: } [0, \infty)$$

	\item An open rectangular box with volume $2 \text{ m}^3$ has a square base.
		Express the surface area of the box as a function of the length of a side of the base.

		$$hx^2 = 2$$
		$$h = \frac{2}{x^2}$$
		For the surface area we have...
		$$A(x) = 2x^2 + 8xh$$
		Substitute in x for h...
	
		$$A(x) = 2x^2 + 8x\frac{2}{x^2} = 2x^2 + \frac{16}{x} = x^2 + \frac{8}{x}$$

	 	$$\text{Domain: } (0, \infty)$$


	\item A cell phone plan has a basic charge of \$35 a month. The plan includes 400 free minutes
		and charges 10 cents for each additional minute of usage. Write the monthly cost \emph{C}
		as a function of the number of \emph{x} minutes and graph \emph{C} as a function
		of \emph{x} for $0 \leq x \leq 600$.
		
	\item In a certain country, income tax is assessed as follows. There is no tax on income up to \$10,000.
		Any income over \$10,000 is taxed at a rate of 10\%, up to an income of \$20,000. Any income
		over \$20,000 is taxed at 15\%.
		
	\item The function in Example 6 and Exercises 52 and 53(a) are called \emph{step functions} because
		their graphs look like stairs. Give two other examples of step functions that arise in everyday life.
		
		One example of a step function in everyday life is the comversion of marks to grades in an exam
		paper. A second example of a step function is the speed limit along a length of road.
		
	\item[55--56] Graphs of \emph{f} and \emph{g} are shown. Decide whether each function is even, 
		odd, or neither. Explain your reasoning.
		
	\item \emph{g} is even as it is symmetrical about the \emph{y}-axis. \emph{f} is odd as it is symmetrical
		about the origin.
		
	\item \emph{g} is even, as it symmetrical about the \emph{y}-axis. \emph{f} is neither even nor odd, as
		it is symmetrical about a point on the \emph{y}-axis, but not the origin.
		
	\item
		\begin{enumerate}
			\item If the point (5, 3) is on the graph of an even function, what other point must also
				be on the graph?
				
				(-5, 3) must also be on the graph.
				
			\item If the point (5, 3) is on the graph of an odd function. what other point must also
				be on the graph?
				
				(-5, -3) must also be on the graph.
				
		\end{enumerate}

	\item A function \emph{f} has domain [--5, 5] and a portion of its graph is shown.
		\begin{enumerate}
			\item Complete the graph of \emph{f} if it is known and \emph{f} is even.
			
			\item Complete the graph of \emph{f} if it is known and \emph{f} is odd.
		\end{enumerate}
		
	\item[59--64] Determine whether \emph{f} is even, odd or neither. If you have a graphing calculator,
		use it to check your answer visually.
		
		\item $f(x) = \frac{x}{x^2 + 1}$
		
			$$f(-x) = \frac{-x}{(-x)^2 + 1}$$
			$$f(-x) = - \frac{x}{x^2 + 1}$$
			
			This function is odd.

	\item $f(x) = \frac{x^2}{x^4 + 1}$

		$$f(-x) = \frac{(-x)^2}{(-x)^4 + 1} = \frac{x^2}{x^4 + 1}$$

		This function is \emph{even}.

	\item $f(x) = \frac{x}{x + 1}$

		$$f(-x) = \frac{-x}{1 - x}$$

		This function is neither odd not even.

	\item $f(x) = x | x |$

		$$f(-x) = (-x) | (-x) | =  - (x | x |)$$

		This function is \emph{odd}.

	\item $f(x) = 1 + 3x^2 - x^4$

		$$f( -x) = 1 + 3(-x)^2 - (-x)^4 = 1 + 3x^2 - x^4$$

		This function is \emph{even}.

	\item $f(x) = 1 + 3x^2- x^5$

		$$f( - x ) = 1 + 3(-x)^2 - (-x)^5 = 1 - 3x^2 + x^5$$

		This function is neither even nor odd.

	\item If \emph{f} and \emph{g} are both even functions, is $f + g$ even? If
		\emph{f} and \emph{g} are both odd functions. is $f + g$ odd? What if
		\emph{f} is even and \emph{g} is odd? Justify your answers.

		If $f$ and $g$ are both even functions then...
		$$ f(-x) = f(x), g(-x) = g(x)$$
		$$h(-x) = f(-x) + g(-x) = f(x) + g(x) = h(x)$$
		So $f + g$ is even when \emph{f} and \emph{g} are even.

		If $g$ and $f$ are both odd functions then...
		$$f(-x) = - f(x), g(-x) = -g(x)$$
		$$h(-x) = f(-x) + g(-x) = - (f(x) + g(x)) = - h(x)$$
		So $f + g$ is odd when $f$ and $g$ are odd.

		If $f$ is even and $g$ is odd then,
		
		$$ f(-x) = f(x), g(-x) = -g(x)$$
		$$h(-x) = f(-x) + g(-x) = f(x) - g(x)$$

		So in this case $f + g$ is neither.
		
	\item If \emph{f} and \emph{g} are both even functions, is the product $fg$ even? If
		\emph{f} and \emph{g} are both odd functions, is $fg$ odd? What if \emph{f} is even
		and \emph{g} is odd? Justify your answers.
		
		If $f$ and $g$ are both even functions then...
		$$ f(-x) = f(x), g(-x) = g(x)$$
		$$h(-x) = f(-x) \times g(-x) = f(x) \times g(x) = h(x)$$
		So $fg$ is even when \emph{f} and \emph{g} are even.
		
		If $g$ and $f$ are both odd functions then...
		$$f(-x) = - f(x), g(-x) = -g(x)$$
		$$h(-x) = f(-x) \times g(-x) = [-f(x)] \times [-g(x)] = f(x) \times g(x) =  h(x)$$
		So $fg$ is \emph{even} when $f$ and $g$ are odd.
		
		If $f$ is even and $g$ is odd then,
		
		$$ f(-x) = f(x), g(-x) = -g(x)$$
		$$h(-x) = f(-x) \times g(-x) = [-f(x)] \times g(x) = - f(x) g(x) = - h(x)$$

		So in this case $fg$ is odd.
		
	\end{enumerate}
	

\end{document}