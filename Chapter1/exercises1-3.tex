\documentclass{article}

\renewcommand{\thesection}{} % remove section
\renewcommand{\labelenumiii}{(\roman{enumiii})}


\usepackage{amsmath}

\begin{document}
	\section{1.3 Exercises}
	
	\begin{enumerate}
		\item If a ball is thrown into the air with a velocity of 10 m/s, its height in metres
			\emph{t} seconds later is given by $y = 10 - 4.9t^2$.
			
			\begin{enumerate}
				\item Find the average velocity for the time period beginning when $t = 1.5$ and
					lasting
				\begin{enumerate}
					\item 0.5 second
					
					Difference quotient:
					$$\frac{10(2) - 4.9(2)^2 - (10(1.5) - 4.9(1.5)^2)}{0.5}$$
					$$\frac{0.4 - 3.975}{0.5}$$
					$$\frac{-3.575}{0.5} = -7.15 \text{m/s}$$
					
					\item 0.1 second
					
					Difference quotient:
					$$\frac{10(1.6) - 4.9(1.6)^2 - (10(1.5) - 4.9(1.5)^2)}{0.1}$$
					$$\frac{3.456 - 3.975)}{0.1}$$
					$$\frac{-0.519}{0.1} = -5.19 \text{m/s}$$
					
					\item 0.05 second
					
					Difference quotient
					
					$$\frac{10(1.55) - 4.9(1.55)^2 -  (10(1.5) - 4.9(1.5)^2)}{0.05}$$
					$$\frac{ 3.72775 - 3.975}{0.05}$$
					$$\frac{ -0.24725}{0.05} = - 4.945 \text{m/s}$$
					
					\item 0.01 second
					
					Difference quotient
					
					$$\frac{10(1.51) - 4.9(1.51)^2 -  (10(1.5) - 4.9(1.5)^2)}{0.01}$$
					$$\frac{ 3.92751 - 3.975 }{0.01}$$
					$$\frac{ -0.04749}{0.01} = - 4.749\text{m/s}$$					
					
				\end{enumerate}
				\item Estimate the instantaneous velocity when $t = 1.5$.
				
					From the results from part (a) it looks like the instantaneous velocity is converging on -- 4.7 m/s.
				
				
			\end{enumerate}
			
		\item If an arrow is shot upward on the moon with a velocity of 58 m/s, its height in metres \emph{t} seconds
			later is given by $h = 58t - 0.83t^2$.
			
			\begin{enumerate}
			
				\item Find the average velocity over the given time intervals:
				
				\begin{enumerate}
					\item $[1, 2]$
					
						Difference quotient
						
						$$\frac{58(2) - 0.83(2)^2 - (58(1) - 0.83(1)^2)}{2 - 1}$$
						$$\frac{116 - 3.32 - (58 - 0.83)}{1}$$
						$$112.68 - 57.17 = 55.51$$
						
					\item $[1, 1.5]$
					
						Difference quotient
						
						$$\frac{58(1.5) - 0.83(1.5)^2 - (58(1) - 0.83(1)^2)}{1.5 - 1}$$
						$$\frac{87 - 0.83(2.25) - (58 - 0.83)}{0.5}$$
						$$\frac{87 - 1.8675 - 57.17}{0.5}$$
						$$\frac{27.9625}{0.5} = 55.925$$
						
					\item $[1, 1.1]$
					
						Difference quotient
						
						$$\frac{58(1.1) - 0.83(1.1)^2 - (58(1) - 0.83(1)^2)}{1.1 - 1}$$
						$$\frac{63.8 - 0.83(1.21) - (58 - 0.83)}{0.1}$$
						$$\frac{63.8 - 1.0043 - 57.17}{0.1}$$
						$$\frac{5.6257}{0.1} = 56.257$$
						
					\item $[1, 1.01]$
					
						Difference quotient
						
						$$\frac{58(1.01) - 0.83(1.01)^2 - (58(1) - 0.83(1)^2)}{1.01 - 1}$$
						$$\frac{58.58 - 0.83(1.0201) - (58 - 0.83)}{0.01}$$
						$$\frac{58.58 - 0.846683 - 57.17}{0.01}$$
						$$\frac{0.563317}{0.01} = 56.3317$$

					\item $[1, 1.001]$
					
						$$\frac{58(1.001) - 0.83(1.001)^2 -  (58(1) - 0.83(1)^2)}{1.001 - 1}$$
						$$\frac{58.058 - 0.83(1.002001) -  (58 - 0.83)}{0.001}$$
						$$\frac{58.058 - 0.83166083 - 57.17}{0.001}$$
						$$\frac{0.05633917}{0.001} = 56.33917$$
						
				\end{enumerate}
				
				\item Estimate the instantaneous velocity when $t = 1$.
				
					It looks as if the instantaneous velocity is converging on 56.4 m/s.
			\end{enumerate}
			
			\item For the function \emph{f} whose graph is given, state the value of each quantity, if it exists.
				If it does not exist, explain why.
				
				\begin{enumerate}
				
					\item $\lim\limits_{x\to 1} f(x)$
					
						The limit of $f(x)$ as \emph{x} goes to 1 is 2.
						
					\item $\lim\limits_{x \to 3^{-}} f(x)$
					
						The limit of $f(x)$ as \emph{x} goes to 3 from the left is 1.
						
					\item $\lim\limits_{x \to 3^{+}} f(x)$
					
						The limit of $f(x)$ as \emph{x} approaches 3 from the right is 4.
						
					\item $\lim\limits_{x \to 3} f(x)$
					
						The limit of $f(x)$ as \emph{x} does not exist as $\lim\limits_{x \to 3^{-}} f(x)$ is not equal
						to $\lim\limits_{x \to 3^{+}} f(x)$.
						
					\item $f(3)$
					
						$f(3) = 3$
				\end{enumerate}
				
			\item For the function \emph{f} whose graph is given, state the value of each quantity, if it exists. If it does not
				exist, explain why.
				
				\begin{enumerate}
				
					\item $\lim \limits_{x \to 0} f(x)$
					
						The limit of $f(x)$ as \emph{x} approaches 0 is 3.
						
					\item $\lim \limits_{x \to 3^{-}} f(x)$
					
						The limit of $f(x)$ as \emph{x} approaches 3 from the left is 4.
						
					\item $\lim \limits_{x \to 3^{+}} f(x)$
					
						The limit of $f(x)$ as \emph{x} approaches 3 from the right is 2.
						
					\item $\lim \limits_{x \to 3} f(x)$
					
						The limit of $f(x)$ as \emph{x} approaches 3 does not exist as 
						$\lim\limits_{x \to 3^{-}} f(x)$ is not equal to $\lim\limits_{x \to 3^{+}} f(x)$.
						
					\item $f(3)$
					
						$f(3) = 3$
						
				\end{enumerate}
				
			\item For the function \emph{g} whose graph is given, state the value of each quantity, if it exists. If it does
				not exist explain why.
				
				\begin{enumerate}
				
					\item $\lim \limits_{t \to 0^{-}} g(t)$
					
						The limit of $g(t)$ as \emph{t} approaches 0 from the left is -1.
						
					\item $\lim \limits_{t \to 0^{+}} g(t)$
					
						The limit of $g(t)$ as \emph{t} approaches 0 from the right -2.
						
					\item $\lim \limits_{t \to 0} g(t)$
					
						The limit of $g(t)$ as \emph{t} approaches 0 does not exist as 
						$\lim\limits_{t \to 0^{-}} g(t)$ is not equal to $\lim\limits_{t \to 0^{+}} g(t)$.
						
					\item $\lim \limits_{t \to 2^{-}} g(t)$
					
						The limit of $g(t)$ as \emph{t} approaches 2 from the left is 2.
						
					\item $\lim \limits_{t \to 2^{+}} g(t)$
					
						The limit of $g(t)$ as \emph{t} approaches 2 from the right is 0.
						
					\item $\lim \limits_{t \to 2} g(t)$
					
						The limit of $g(t)$ as \emph{t} approaches 2 does not exist as 
						$\lim\limits_{t \to 2^{-}} g(t)$ is not equal to $\lim\limits_{t \to 2^{+}} g(t)$.
						
					\item $g(2)$
					
						$g(2) = 1$
						
					\item $\lim \limits_{t \to 4} g(t)$
					
						The limit of $g(t)$ as \emph{t} approaches 4 is 3.
				\end{enumerate}
				
				\item Sketch the graph of the following function and use it to determine the values of
					\emph{a} for which $\lim_{x \to a} f(x)$ exists.
					
 \begin{displaymath}
   f(x) = \left\{
     \begin{array}{ll}
       1 +\sin x & \text{if } x < 0\\
       \cos x &  \text{if } 0 \leq x \leq \pi \\
       \sin x & \text{if } x > \pi 
     \end{array}
   \right.
\end{displaymath} 

			\item[7-10] Sketch the graph of an example of a function \emph{f} that satisfies all of the 
					given conditions.
					
			\item $\lim \limits_{x \to 0^{-}} f(x) = -1, \lim \limits_{x \to 0^{+}}f(x) = 2, f(0) = 1$
	\end{enumerate}
\end{document}