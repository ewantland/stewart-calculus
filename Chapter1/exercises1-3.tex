\documentclass{article}

\renewcommand{\thesection}{} % remove section
\renewcommand{\labelenumiii}{(\roman{enumiii})}


\usepackage{amsmath}

\begin{document}
	\section{1.3 Exercises}
	
	\begin{enumerate}
		\item If a ball is thrown into the air with a velocity of 10 m/s, its height in metres
			\emph{t} seconds later is given by $y = 10 - 4.9t^2$.
			
			\begin{enumerate}
				\item Find the average velocity for the time period beginning when $t = 1.5$ and
					lasting
				\begin{enumerate}
					\item 0.5 second
					
					Difference quotient:
					$$\frac{10(2) - 4.9(2)^2 - (10(1.5) - 4.9(1.5)^2)}{0.5}$$
					$$\frac{0.4 - 3.975}{0.5}$$
					$$\frac{-3.575}{0.5} = -7.15 \text{m/s}$$
					
					\item 0.1 second
					
					Difference quotient:
					$$\frac{10(1.6) - 4.9(1.6)^2 - (10(1.5) - 4.9(1.5)^2)}{0.1}$$
					$$\frac{3.456 - 3.975)}{0.1}$$
					$$\frac{-0.519}{0.1} = -5.19 \text{m/s}$$
					
					\item 0.05 second
					
					Difference quotient
					
					$$\frac{10(1.55) - 4.9(1.55)^2 -  (10(1.5) - 4.9(1.5)^2)}{0.05}$$
					$$\frac{ 3.72775 - 3.975}{0.05}$$
					$$\frac{ -0.24725}{0.05} = - 4.945 \text{m/s}$$
					
					\item 0.01 second
					
					Difference quotient
					
					$$\frac{10(1.51) - 4.9(1.51)^2 -  (10(1.5) - 4.9(1.5)^2)}{0.01}$$
					$$\frac{ 3.92751 - 3.975 }{0.01}$$
					$$\frac{ -0.04749}{0.01} = - 4.749\text{m/s}$$					
					
				\end{enumerate}
				\item Estimate the instantaneous velocity when $t = 1.5$.
				
					From the results from part (a) it looks like the instantaneous velocity is converging on -- 4.7 m/s.
				
				
			\end{enumerate}
			
		\item If an arrow is shot upward on the moon with a velocity of 58 m/s, its height in metres \emph{t} seconds
			later is given by $h = 58t - 0.83t^2$.
			
			\begin{enumerate}
			
				\item Find the average velocity over the given time intervals:
				
				\begin{enumerate}
					\item $[1, 2]$
					
						Difference quotient
						
						$$\frac{58(2) - 0.83(2)^2 - (58(1) - 0.83(1)^2)}{2 - 1}$$
						$$\frac{116 - 3.32 - (58 - 0.83)}{1}$$
						$$112.68 - 57.17 = 55.51$$
						
					\item $[1, 1.5]$
					
						Difference quotient
						
						$$\frac{58(1.5) - 0.83(1.5)^2 - (58(1) - 0.83(1)^2)}{1.5 - 1}$$
						$$\frac{87 - 0.83(2.25) - (58 - 0.83)}{0.5}$$
						$$\frac{87 - 1.8675 - 57.17}{0.5}$$
						$$\frac{27.9625}{0.5} = 55.925$$
						
					\item $[1, 1.1]$
					
						Difference quotient
						
						$$\frac{58(1.1) - 0.83(1.1)^2 - (58(1) - 0.83(1)^2)}{1.1 - 1}$$
						$$\frac{63.8 - 0.83(1.21) - (58 - 0.83)}{0.1}$$
						$$\frac{63.8 - 1.0043 - 57.17}{0.1}$$
						$$\frac{5.6257}{0.1} = 56.257$$
						
					\item $[1, 1.01]$
					
						Difference quotient
						
						$$\frac{58(1.01) - 0.83(1.01)^2 - (58(1) - 0.83(1)^2)}{1.01 - 1}$$
						$$\frac{58.58 - 0.83(1.0201) - (58 - 0.83)}{0.01}$$
						$$\frac{58.58 - 0.846683 - 57.17}{0.01}$$
						$$\frac{0.563317}{0.01} = 56.3317$$

					\item $[1, 1.001]$
					
						$$\frac{58(1.001) - 0.83(1.001)^2 -  (58(1) - 0.83(1)^2)}{1.001 - 1}$$
						$$\frac{58.058 - 0.83(1.002001) -  (58 - 0.83)}{0.001}$$
						$$\frac{58.058 - 0.83166083 - 57.17}{0.001}$$
						$$\frac{0.05633917}{0.001} = 56.33917$$
						
				\end{enumerate}
				
				\item Estimate the instantaneous velocity when $t = 1$.
				
					It looks as if the instantaneous velocity is converging on 56.4 m/s.
			\end{enumerate}
			
			\item For the function \emph{f} whose graph is given, state the value of each quantity, if it exists.
				If it does not exist, explain why.
				
				\begin{enumerate}
				
					\item $\lim\limits_{x\to 1} f(x)$
					
						The limit of $f(x)$ as \emph{x} goes to 1 is 2.
						
					\item $\lim\limits_{x \to 3^{-}} f(x)$
					
						The limit of $f(x)$ as \emph{x} goes to 3 from the left is 1.
						
					\item $\lim\limits_{x \to 3^{+}} f(x)$
					
						The limit of $f(x)$ as \emph{x} approaches 3 from the right is 4.
						
					\item $\lim\limits_{x \to 3} f(x)$
					
						The limit of $f(x)$ as \emph{x} does not exist as $\lim\limits_{x \to 3^{-}} f(x)$ is not equal
						to $\lim\limits_{x \to 3^{+}} f(x)$.
						
					\item $f(3)$
					
						$f(3) = 3$
				\end{enumerate}
				
			\item For the function \emph{f} whose graph is given, state the value of each quantity, if it exists. If it does not
				exist, explain why.
				
				\begin{enumerate}
				
					\item $\lim \limits_{x \to 0} f(x)$
					
						The limit of $f(x)$ as \emph{x} approaches 0 is 3.
						
					\item $\lim \limits_{x \to 3^{-}} f(x)$
					
						The limit of $f(x)$ as \emph{x} approaches 3 from the left is 4.
						
					\item $\lim \limits_{x \to 3^{+}} f(x)$
					
						The limit of $f(x)$ as \emph{x} approaches 3 from the right is 2.
						
					\item $\lim \limits_{x \to 3} f(x)$
					
						The limit of $f(x)$ as \emph{x} approaches 3 does not exist as 
						$\lim\limits_{x \to 3^{-}} f(x)$ is not equal to $\lim\limits_{x \to 3^{+}} f(x)$.
						
					\item $f(3)$
					
						$f(3) = 3$
						
				\end{enumerate}
				
			\item For the function \emph{g} whose graph is given, state the value of each quantity, if it exists. If it does
				not exist explain why.
				
				\begin{enumerate}
				
					\item $\lim \limits_{t \to 0^{-}} g(t)$
					
						The limit of $g(t)$ as \emph{t} approaches 0 from the left is -1.
						
					\item $\lim \limits_{t \to 0^{+}} g(t)$
					
						The limit of $g(t)$ as \emph{t} approaches 0 from the right -2.
						
					\item $\lim \limits_{t \to 0} g(t)$
					
						The limit of $g(t)$ as \emph{t} approaches 0 does not exist as 
						$\lim\limits_{t \to 0^{-}} g(t)$ is not equal to $\lim\limits_{t \to 0^{+}} g(t)$.
						
					\item $\lim \limits_{t \to 2^{-}} g(t)$
					
						The limit of $g(t)$ as \emph{t} approaches 2 from the left is 2.
						
					\item $\lim \limits_{t \to 2^{+}} g(t)$
					
						The limit of $g(t)$ as \emph{t} approaches 2 from the right is 0.
						
					\item $\lim \limits_{t \to 2} g(t)$
					
						The limit of $g(t)$ as \emph{t} approaches 2 does not exist as 
						$\lim\limits_{t \to 2^{-}} g(t)$ is not equal to $\lim\limits_{t \to 2^{+}} g(t)$.
						
					\item $g(2)$
					
						$g(2) = 1$
						
					\item $\lim \limits_{t \to 4} g(t)$
					
						The limit of $g(t)$ as \emph{t} approaches 4 is 3.
				\end{enumerate}
				

				\item Sketch the graph of the following function and use it to determine the values of
					\emph{a} for which $\lim_{x \to a} f(x)$ exists.
					
 \begin{displaymath}
   f(x) = \left\{
     \begin{array}{ll}
       1 +\sin x & \text{if } x < 0\\
       \cos x &  \text{if } 0 \leq x \leq \pi \\
       \sin x & \text{if } x > \pi 
     \end{array}
   \right.
\end{displaymath} 

			\item[7-10] Sketch the graph of an example of a function \emph{f} that satisfies all of the 
					given conditions.
					
			\item $\lim \limits_{x \to 0^{-}} f(x) = -1, \lim \limits_{x \to 0^{+}}f(x) = 2, f(0) = 1$
			
			\item $\lim \limits _{x \to 0^{-}} f(x) = 1, \lim \limits _{x \to 0^{+}} f(x) = -1, \lim \limits _{x \to 2^{-}} f(x)=0,
				\lim \limits _{x \to 2^{+}} f(x)  = 1, f(2) = 1, f(0) \text{ is undefined}$
				
			\item $\lim \limits _{x \to 3^{+}} f(x) = 4, \lim \limits _{x \to 3^{-}} f(x) = 2, \lim \limits _{x \to -2} f(x) = 2,
				f(3) = 3, f(-1) = 1$
				
			\item $\lim \limits_{x \to 0^{-}} f(x) = 2, \lim \limits _{x \to 0^{+}} f(x) = 0, \lim \limits _{x \to 4^{-}} = 3,
				\lim \limits _{x \to 4^{+}} = 0, f(0) = 2, f(4) = 1$

			\item[11--14] Guess the value of the liimit (if it exists) by evaluating the function at the given numbers (correct to six decimal places).
			
			\item $\lim \limits_{x \to 2} \frac{x^2 - 2x}{x^2 - x - 2}$ 
			
			  \emph{x} = 2.5, 2.1, 2.05, 2.01, 2.005 , 2.001, 1.9, 1.95, 1.99, 1.995, 1.999
			  
			  $$\frac{x^2 - 2x}{x^2 - x - 2} = \frac{x(x - 2)}{(x + 1)(x - 2)}$$
			  
			  $$f(2.5) = \frac{2.5(2.5 - 2)}{(2.5 + 1)(2.5 - 2)}$$
			  
			  $$f(2.5) = \frac{2.5(0.5)}{(3.5)(0.5)} = \frac{1.25}{1.75} = 0.71428571428$$
			  
			  $$f(2.1) = \frac{2.1(2.1 - 2)}{(2.1 + 1)(2.1 - 2)}$$
			  $$f(2.1) = \frac{2.1(0.1)}{(3.1)(0.1)} = \frac{0.21}{0.31} = 0.67741935483$$
			  
			  $$f(2.05) = \frac{2.05(2.05 - 2)}{(2.05 + 1)(2.05 - 2)}$$
			  $$f(2.05) = \frac{2.05(0.05)}{3.05(0.05)} = \frac{0.1025}{0.1525} = 0.67213114754$$
			  
			  $$f(2.01) = \frac{2.01(2.01 - 2)}{(2.01 + 1)(2.01 - 2)}$$
			  $$f(2.01) = \frac{2.01(0.01)}{3.01(0.01)} = \frac{0.0201}{0.0301} = 0.66777408637$$
			  
			  $$f(2.005) = \frac{2.005(2.005 - 2)}{(2.005 + 1)(2.005 - 2)} $$ 
			  $$f(2.005) = \frac{2.005(0.005)}{3.005(0.005)} = \frac{2.005}{3.005} = 0.66722129783$$
			  
			  $$f(2.001) = \frac{2.001(2.001 - 2)}{(2.001 + 1)(2.001 - 2)}$$
			  $$f(2.001) = \frac{2.001}{3.001} = 0.66677774075$$
			  
			  $$f(1.9) = \frac{1.9(1.9 - 2)}{(1.9 + 1)(1.9 - 2)}$$
			  $$f(1.9) = \frac{1.9}{2.9} = 0.65517241379$$
			  
			  $$f(1.95) = \frac{1.95(1.95 - 2)}{(1.95 + 1)(1.95 - 2)}$$
			  $$f(1.95) = \frac{1.95}{2.95} = 0.66101694915$$
			  
			  $$f(1.99) = \frac{1.99(1.99 - 2)}{(1.99 + 1)(1.99 - 2)}$$
			  $$f(1.99) = \frac{1.99}{2.99} = 0.66555183946$$
			  
			  $$f(1.995) = \frac{1.995(1.995 - 2)}{(1.995 + 1)(1.995 - 2)}$$
			  $$f(1.995) = \frac{1.995}{2.995} = 0.66611018363$$
			  
			  $$f(1.999) = \frac{1.999(1.999 - 2)}{(1.999 + 1)(1.999 - 2)}$$
			  $$f(1.999) = \frac{1.999}{2.999} = 0.6665555185$$
\begin{center}
			  \begin{tabular}{|c|c|}
			  	\hline
			  	$x$ & $f(x)$ \\
			  	\hline \hline			  	
			  	2.5 & 0.714286 \\
			  	2.1 &  0.677419 \\
			  	2.05 &  0.672131 \\
			  	2.01 &  0.667774 \\
			  	2.005 &  0.667221 \\
			  	2.001 & 0.666778 \\
			  	1.999 &  0.666556 \\
			  	1.995 & 0.666110 \\
			  	1.99 & 0.665552 \\
			  	1.95 &  0.661017 \\
			  	1.9 &  0.655172 \\
			  	\hline
			  \end{tabular}
\end{center}

			\item $\lim \limits_{x \to -1} \frac{x^2 - 2x}{x^2 - x - 2}$
			
				\emph{x} = 0, -0.5, -0.9, -0.95, -0.99, -0.999, -2, -1.5, -1.1, -1.01, -1.001
				
				$$\frac{x^2 - 2x}{x^2 - x - 2} = \frac{x(x-2)}{(x + 1)(x - 2)} = \frac{x}{x+1}$$
				
				$$f(0) = \frac{0}{0 + 1} = 0$$
				
				$$f(-0.5) = \frac{-0.5}{1 - 0.5} = \frac{-0.5}{0.5} = -1$$
				
				$$f(-0.9) = \frac{-0.9}{1 - 0.9} = \frac{-0.9}{0.1} = -9$$
				
				$$f(-0.95) = \frac{-0.95}{1 - 0.95} = \frac{-0.95}{0.05} = -19$$
				
				$$f(-0.99) = \frac{-0.99}{1- 0.99} = \frac{-0.99}{0.01} = -99$$
				
				$$f(-0.999) = \frac{-0.999}{1 - 0.999} = \frac{-0.999}{0.001} = -999$$
				
				$$f(-2) = \frac{-2}{1-2} = \frac{-2}{-1} = 2$$
				
				$$f(-1.5)  = \frac{-1.5}{1 - 1.5} = \frac{-1.5}{-0.5} = 3$$
				
				$$f(-1.1) = \frac{-1.1}{1 - 1.1} = \frac{-1.1}{-0.1} = 11$$
				
				$$f(-1.01) = \frac{-1.01}{1 - 1.01} = \frac{-1.01}{-0.01} = 101$$
				
				$$f(-1.001) = \frac{-1.001}{1-1.001} = \frac{-1.001}{-0.001} = 1001$$
				
				\begin{center}
				\begin{tabular}{|c|c|}
					\hline
					$x$ & $f(x)$ \\
					\hline \hline
					0 & 0 \\
					-0.5 & -1 \\
					-0.9 & -9 \\
					-0.95 & -19 \\
					-0.99 & -99 \\
					-0.999 & -999 \\
					-1.001 & 1001 \\
					-1.01 & 101 \\
					-1.1 & 11 \\
					-1.5 & 3 \\
					-2 & 2 \\
					\hline
				\end{tabular}
				\end{center}
				
				The limits are $\lim \limits _{x \to -1^{+}}  \frac{x^2 - 2x}{x^2 - x - 2} = -\infty$ and
				$\lim \limits _{x \to -1^{-}} \frac{x^2 - 2x}{x^2 - x - 2} = +\infty$
				
			\item $\lim \limits _{x \to 0} \frac{\sin x}{x + \tan x}$
			
				$x = \pm 1, \pm 0.5, \pm 0.2, \pm 0.1, \pm 0.05, \pm 0.01$
				
				$$f(1) = \frac{\sin 1}{1 + \tan 1} = 0.329033$$
				$$f(0.5) = \frac{\sin 0.5}{0.5 + \tan 0.5} = 0.458209$$
				$$f(0.2) = \frac{\sin 0.2}{0.2 + \tan 0.2} = 0.493331$$
				$$f(0.1) = \frac{\sin 0.1}{0.1 + \tan 0.1} = 0.498333$$
				$$f(0.05) = \frac{\sin 0.05}{0.05 + \tan 0.05} = 0.499583$$
				$$f(0.01) = \frac{\sin 0.01}{0.01 + \tan 0.01} = 0.499983$$
				$$f(-0.01) = \frac{\sin -0.01}{-0.01 + \tan -0.01} = 0.499983$$
				$$f(-0.05) = \frac{\sin -0.05}{-0.05 + \tan -0.05} = 0.499583$$
				$$f(-0.1) = \frac{\sin - 0.1} {-0.1 + \tan -0.1} = 0.498333$$
				$$f(-0.2) = \frac{\sin -0.2}{-0.2 + \tan -0.2} = 0.493331$$
				$$f(-0.5) = \frac{\sin -0.5}{-0.5 + \tan -0.5} = 0.458209$$
				$$f(-1) = \frac{\sin -1}{-1 + \tan -1} =0.329033$$
				
				\begin{center}
				\begin{tabular}{|c|c|}
				\hline
				$x$ & $f(x)$ \\
				\hline \hline
				1 &  0.32903 \\
				0.5 & 0.458209 \\
				0.2 &  0.493331 \\
				0.1 &  0.498333 \\
				0.05 & 0.499583 \\
				0.01 &  0.499983 \\
				-0.01 & 0.499983 \\
				-0.05 & 0.499583\\
				-0.1 &  0.498333 \\
				-0.2 &  0.493331 \\
				-0.5 & 0.458209 \\
				-1 & 0.329033 \\
				\hline
				\end{tabular}
				\end{center}
				
				It looks like $\lim \limits _{x \to 0} \frac{\sin x}{x + \tan x} = 0.5$
			
			\item 	$\lim \limits _{h \to 0} \frac{(2 + h)^5 - 32}{h}$
			
			$h = \pm 0.5, \pm 0.1, \pm 0.01, \pm 0.001, \pm 0.0001$
			
			$$f(0.5) = \frac{(2 + 0.5)^5 -32}{0.5} = \frac{2.5^5 - 32}{0.5} $$
			$$= \frac{97.65625 - 32}{0.5} = \frac{65.65625}{0.5} = 131.3125$$
			
			$$f(0.1) = \frac{(2 + 0.1)^5 - 32}{0.1} = \frac{2.1^5 -32}{0.1} $$
			$$ = \frac{40.84101 - 32}{0.1} = \frac{8.84101}{0.1} = 88.4101$$
			
			$$f(0.01) = \frac{(2 + 0.01)^5 - 32}{0.01} = \frac{2.01^5 - 32}{0.01} = \frac{32.8080401001 - 32}{0.01} $$
			$$= \frac{0.8080401001}{0.01} = 80.80401001$$
			
			$$f(0.001) = \frac{(2+ 0.001)^5 - 32}{0.001} = \frac{2.001^5 - 32}{0.001} = $$
			$$\frac{32.080080040010001 - 32}{0.001} = \frac{0.080080040010001}{0.001} = 80.080040010001$$
			
			$$f(0.0001) = \frac{(2 + 0.0001)^5 - 32}{0.0001} = \frac{2.0001^5  - 32}{0.0001} $$
			$$= \frac{32.008000800040001 - 32}{0.0001} $$
			$$= \frac{0.008000800040001}{0.0001} = 80.00800040001$$
			
			$$f(-0.0001) = \frac{(2 - 0.0001)^5 - 32}{-0.0001}  = \frac{1.9999^5 - 32}{-0.0001}$$ 
			$$= \frac{31.992000799960001-32}{-0.0001}$$
			$$ = \frac{-0.007999200039999}{-0.0001} = 79.99200039999$$
			
			$$f(-0.001) = \frac{(2 - 0.001)^5 - 32}{-0.001}  =\frac{1.999^5 - 32}{-0.001} $$
			$$= \frac{31.920079960009999 - 32}{-0.001}$$
			$$ = \frac{-0.079920039990001}{-0.001} = 79.920039990001$$
			
			$$f(-0.01) = \frac{(2-0.01)^5 - 32}{-0.01} = \frac{1.99^5 - 32}{-0.01} = \frac{31.2079600999 - 32}{-0.01} = $$
			$$\frac{-0.7920399001}{-0.01} = 79.20399001$$
			
			$$f(-0.1) = \frac{(2- 0.1)^5 - 32}{-0.1} = \frac{1.9^5 - 32}{-0.1} = \frac{24.76099 -32}{-0.1} $$
			$$= \frac{-7.23901}{-0.1} = 72.3901$$
			
			$$f(-0.5) = \frac{(2- 0.5)^2 - 32}{-0.5} = \frac{1.5^5 -32}{-0.5} = \frac{7.59375 - 32}{-0.5} $$
			$$= \frac{-24.40625}{-0.5} = 48.8125$$
			
			\begin{center}
				\begin{tabular}{|c|c|}
				\hline
				$x$ & $f(x)$ \\
				\hline \hline
				0.5 &  131.3125 \\
				0.1 & 88.4101 \\
				0.01 &  80.804010 \\
				0.001 &   80.080040 \\
				0.0001 & 80.008000 \\
				-0.0001 &  79.992000 \\
				-0.001 & 79.920040 \\
				-0.01 & 79.203990 \\
				-0.1 &  72.3901 \\
				-0.5 & 48.8125 \\
				\hline
				\end{tabular}
			\end{center}
			
			It looks like $\lim \limits _{h \to 0} \frac{(2 + h)^5 - 32}{h} = 80$
			
			\item[15--18] Use a table of values to estimate the value of the limit. If you have a graphing
				device, use it to confirm your result graphically.
				
			\item $\lim \limits _{x \to 0} \frac{\sqrt{x + 4} - 2}{x}$
			
				$$f(1) = \frac{\sqrt{1 + 4} - 2}{1} = \sqrt{5} - 2 = $$
				$$2.2360679774997897 - 2 = 0.2360679774997897$$
				
				$$f(0.5) = \frac{\sqrt{0.5 + 4} - 2}{0.5} =\frac{\sqrt{4.5} - 2}{0.5}$$
				$$= \frac{2.1213203435596426 - 2}{0.5} = \frac{0.1213203435596426}{0.5}$$
				$$ = 0.2426406871192852$$
				
				$$f(0.1) = \frac{\sqrt{0.1 + 4} - 2}{0.1} = \frac{\sqrt{4.1} - 2}{0.1}$$
				$$ = \frac{2.0248456731316587 - 2}{0.1} = \frac{0.0248456731316587}{0.1}$$
				$$ = 0.248456731316587$$
				
				$$f(0.01) = \frac{\sqrt{0.01 + 4}-2}{0.01} = \frac{\sqrt{4.01} - 2}{0.01}$$
				$$= \frac{2.0024984394500786 - 2}{0.01} = \frac{0.0024984394500786}{0.01}$$
				$$ = 0.24984394500786$$
				
				$$f(0.001) = \frac{\sqrt{0.001 + 4}-2}{0.001} = \frac{\sqrt{4.001} -2}{0.001}$$
				$$= \frac{2.0002499843769528-2}{0.001} = \frac{0.0002499843769528}{0.001}$$
				$$ = 0.2499843769528$$
				
				$$f(-0.001) = \frac{\sqrt{4 - 0.001} - 2}{-0.001}  =\frac{\sqrt{3.999} - 2}{-0.001}$$
				$$ = \frac{1.9997499843730466 - 2}{-0.001} = \frac{-0.0002500156269534}{-0.001}$$
				$$ = 0.2500156269534$$
				
				$$f(-0.01) = \frac{\sqrt{4 - 0.01} - 2}{-0.01} = \frac{\sqrt{3.99} - 2}{-0.01}$$
				$$= \frac{1.9974984355438179 - 2}{-0.01} = \frac{-0.0025015644561821}{-0.01}$$
				$$ = 0.25015644561821$$
				
			\begin{center}
				\begin{tabular}{|c|c|}
				\hline
				$x$ & $f(x)$ \\
				\hline \hline
				1 &  0.2360679774997897 \\
				0.5 & 0.2426406871192852 \\
				0.1 &   0.248456731316587 \\
				0.01 &  0.24984394500786 \\
				0.001 & 0.2499843769528 \\
				-0.001 &  0.2500156269534 \\
				-0.01 & 0.25015644561821 \\
				\hline
				\end{tabular}
			\end{center}
			
			It looks like $\lim \limits _{x \to 0} \frac{\sqrt{x + 4} - 2}{x} = 0.25$
			
			\item $\lim \limits _{x \to 0} \frac{\tan 3x}{\tan 5x}$
			
			$$f(1) = \frac{\tan 3}{\tan 5} = \frac{-0.142546543074}{-3.380515006247} = 0.0421671085058289$$
			
			$$f(0.5) = \frac{\tan 3(0.5)}{\tan 5(0.5)} = \frac{\tan 1.5}{\tan 2.5} $$
			$$= \frac{14.101419947172}{-0.747022297239} = -18.8768394187040382$$
			
			$$f(0.1) = \frac{\tan 3(0.1)}{\tan 5(0.1)} = \frac{\tan 0.3}{\tan 0.5} $$
			$$ = \frac{0.30933624961}{0.546302489844} = 0.5662362067914661$$
			
			$$f(0.01) = \frac{\tan 3(0.01)}{\tan 5(0.01)} = \frac{\tan 0.03}{\tan 0.05}$$
			$$ = \frac{0.030009003241}{0.546302489844} = 0.5996798313822618$$
			
			$$f(0.001) = \frac{\tan 3(0.001)}{\tan 5(0.001} = \frac{\tan 0.003}{\tan 0.005}$$
			$$ = \frac{0.003000009}{0.005000041667} = 0.599996799986667$$
			
			$$f(-0.001) = \frac{\tan 3(-0.001)}{\tan 5(-0.001)} = \frac{\tan -0.003}{\tan -0.005}$$
			$$ = \frac{-0.003000009}{-0.005000041667} = 0.599996799986667$$
			
			$$f(-0.01) = \frac{\tan 3(-0.01)}{\tan 5(-0.01)} = \frac{\tan -0.03}{\tan -0.05}$$
			$$ = \frac{-0.030009003241}{-0.050041708376} = 0.5996798313822618$$
			
			$$f(-0.1) = \frac{\tan 3(-0.1)}{\tan 5(-0.1)} = \frac{\tan -0.3}{\tan -0.5}$$
			$$ = \frac{-0.30933624961}{-0.546302489844} = 0.5662362067914661$$
			
			\begin{center}
				\begin{tabular}{|c|c|}
				\hline
				$x$ & $f(x)$ \\
				\hline \hline
				1 &  0.0421671085058289 \\
				0.5 & -18.8768394187040382 \\
				0.1 &   0.5662362067914661 \\
				0.01 &  0.5996798313822618 \\
				0.001 & 0.599996799986667 \\
				-0.001 &  0.599996799986667 \\
				-0.01 & 0.5996798313822618 \\
				-0.1 & 0.5662362067914661 \\
				\hline
				\end{tabular}
			\end{center}
			
			It looks like $\lim \limits _{x \to 0} \frac{\tan 3x}{\tan 5x} = 0.6$
			
		\item $\lim \limits _{x \to 1} \frac{x^6 - 1}{x^{10} - 1}$
		
			$$f(0) = \frac{(0)^6 - 1}{(0)^10 - 1} = \frac{-1}{-1} = 1$$
			
			$$f(0.5) = \frac{(0.5)^6 - 1}{(0.5)^10 - 1} = \frac{0.015625 - 1}{0.0009765625 - 1}$$
			$$ = \frac{-0.984375}{-0.9990234375} = 0.9853372434017595$$
			
			$$f(0.9) = \frac{(0.9)^6 - 1}{(0.9)^10 - 1} = \frac{0.531441 - 1}{0.3486784401 - 1}$$
			$$ = \frac{-0.468559}{-0.6513215599} = 0.7193973435670389$$
			
			$$f(0.99) = \frac{(0.99)^6 - 1}{(0.99^{10} - 1} = \frac{0.941480149401 - 1}{0.9043820750088045 - 1}$$
			$$ = \frac{-0.058519850599}{-0.0956179249911955} = 0.6120175752024372913$$
			
			$$f(0.999) = \frac{(0.999)^6 - 1}{(0.999)^{10} - 1} = \frac{0.994014980014994 - 1}{0.9900448802097482 - 1}$$
			$$ = \frac{-0.005985019985006}{-0.0099551197902518} = 0.6012001975974834342$$
			
			$$f(1.001) = \frac{(1.001^6 - 1}{1.001^{10} - 1} = \frac{1.006015020015006 - 1}{1.0100451202102522 - 1}$$
			$$ = \frac{0.006015020015006}{0.0100451202102522} = 0.5988002023974765942$$
			
			$$f(1.01) = \frac{1.01^6 - 1}{1.01^{10} - 1} = \frac{1.061520150601 - 1}{1.1046221254112045 - 1}$$
			$$ = \frac{0.061520150601}{0.1046221254112045} = 0.5880223744184373386$$
			
			$$f(1.1) = \frac{1.1^6 - 1}{1.1^{10} - 1} = \frac{1.771561 - 1}{2.5937424601 - 1}$$
			$$ = \frac{0.771561}{1.5937424601} = 0.4841189962094554$$
			
			\begin{center}
				\begin{tabular}{|c|c|}
				\hline
				$x$ & $f(x)$ \\
				\hline \hline
				0 & 1 \\
				0.5 & 0.9853372434017595 \\
				0.9 & 0.7193973435670389 \\
				0.99 & 0.6120175752024372913 \\
				0.999 & 0.6012001975974834342 \\
				1.001 & 0.5988002023974765942 \\
				1.01 & 0.5880223744184373386 \\
				1.1 & 0.4841189962094554 \\
				\hline
				\end{tabular}
			\end{center}
			
			It looks like $\lim \limits _{x \to 1} \frac{x^6 - 1}{x^{10} - 1} = 0.6$
			
		\item $\lim \limits _{x \to 0} \frac{9^x - 5^x}{x}$
		
			$$f(-0.5) = \frac{9^{-0.5} - 5^{-0.5}}{-0.5} = \frac{0.3333333333333333 - 0.4472135954999579}{-0.5}$$
			$$ = \frac{-0.1138802621666246}{-0.5} = 0.2277605243332492$$
			
			$$f(-0.1) = \frac{9^{-0.1} - 5^{-0.1}}{-0.1} = \frac{0.8027415617602307 - 0.8513399225207846}{-0.1}$$
			$$ = \frac{-0.0485983607605539}{-0.1} = 0.485983607605539$$
			
			$$f(-0.01) = \frac{9^{-0.01} - 5^{-0.01}}{-0.01} $$
			$$= \frac{0.9782673857291712 - 0.9840344433634576}{-0.01}$$
			$$ = \frac{-0.0057670576342864}{-0.01} = 0.57670576342864$$
			
			$$f(0.01) = \frac{9^{0.01} - 5^{0.01}}{0.01} = \frac{1.022215413278477 - 1.0162245912673256}{0.01}$$
			$$ = \frac{0.0059908220111514}{0.01} = 0.59908220111514$$
			
			$$f(0.001) = \frac{9^{0.001} - 5^{0.001}}{0.001} = \frac{1.002199640244188 - 1.0016107337527293}{0.001}$$
			$$ = \frac{0.0005889064914587}{0.001} = 0.5889064914587$$
			
			$$f(0.0001) = \frac{9^{0.0001} - 5^{0.0001}}{0.0001}$$
			$$ = \frac{1.0002197465984809 - 1.0001609567433902}{0.001}$$
			$$ = \frac{0.0000587898550907}{0.0001} = 0.587898550907$$
			
			$$f(-0.001) = \frac{9^{-0.001} - 5^{-0.001}}{-0.001} $$
			$$= \frac{0.9978051875535975 - 0.9983918565382238}{-0.001}$$
			$$ = \frac{-0.0005866689846263}{-0.001} = 0.5866689846263$$
			
			\begin{center}
				\begin{tabular}{|c|c|}
				\hline
				$x$ & $f(x)$ \\
				\hline \hline
				-0.5 & 0.2277605243332492 \\
				-0.1 & 0.485983607605539 \\
				-0.01 & 0.57670576342864 \\
				-0.001 &  0.5866689846263 \\
				0.001 & 0.5889064914587 \\
				0.01 &  0.59908220111514 \\
				\hline
				\end{tabular}
			\end{center}
			
			It looks like $\lim \limits _{x \to 0} \frac{9^x - 5^x}{x} = 0.588$
			
		\item 
			\begin{enumerate}
				\item By graphing the function $f(x) = (\cos 2x - \cos x) / x^2$ and zooming in toward the point
					where the graph crosses the \emph{y}-axis, estimate the value of $\lim \limits _{x \to 0} f(x)$.
					
				\item Check your answer in part (a) by evaluating $f(x)$ for values of $x$ that approach 0.
			\end{enumerate}
		
		\item
			\begin{enumerate}
				\item Estimate the value of
				
					$$\lim \limits _{x \to 0} \frac{\sin x}{\sin \pi x}$$
					
					by graphing the function $f(x) = (\sin x)(\sin \pi x)$. State your answer correct to
					two decimal places.
					
				\item Check your answer in part (a) by evaluating  $f(x)$ for values of $x$ that approach 0.
			\end{enumerate}
			
		\item 
			\begin{enumerate}
				\item Evaluate the function $f(x) = x^2 - (2^x/1000)$ for 
				
				$x = 1, 0.8, 0.6, 0.4, 0.2, 0.1, \text{ and } 0.05$, and guess the value of
				
					$$\lim \limits _{x \to 0} (x^2 - \frac{2^x}{1000})$$
					
				$f(1) = (1^2 - \frac{2^1}{1000})$ 
				$= 1 - \frac{2}{1000} = 1 - 0.002 = 0.998$
				
				$f(0.8) = (0.8^2 - \frac{2^{0.8}}{1000}) $
				
				$ = 0.64 - \frac{1.7411011265922483}{1000}$
				$= 0.6382588988734077517$
				
				$f(0.6) = (0.6^2 - \frac{2^{0.6}}{1000})$ 
				
				$ = 0.36 - \frac{1.5157165665103981}{1000}$
				$ = 0.3584842834334896$
				
				$f(0.4) = (0.4^2 - \frac{2^{0.4}}{1000}) $ 
				
				$= 0.16 - \frac{1.3195079107728943}{1000}$
				$ = 0.1586804920892271$
				
				$f(0.2) = (0.2^2 - \frac{2^{0.2}}{1000}) = 0.04 - \frac{1.148698354997035}{1000}$
				$ = 0.038851301645003$
				
				$f(0.1) = (0.1^2 - \frac{2^{0.1}}{1000})$
				
				$ = 0.01 - \frac{1.0717734625362932}{1000}$
				$ = 0.0089282265374637$
				
				$f(0.05) = (0.05^2 - \frac{2^{0.05}}{1000} $
				
				$= 0.0025 - \frac{1.0352649238413775}{1000}$
				$= 0.0014647350761586$
				
				It looks like $\lim \limits _{x \to 0} (x^2 - \frac{2^x}{1000}) = 0$
					
				\item Evaluate $f(x)$ for $x = 0.04, 0.02, 0.01, 0.005, 0.003, \text{ and } 0.001$. Guess again.
				
				$f(0.04)  = (0.04^2 - \frac{2^{0.04}}{1000} $
				
				$= 0.0016 - \frac{1.0281138266560665}{1000}$
				$ = 0.0005718861733439$
				
				$f(0.02) = (0.02^2 - \frac{2^{0.02}}{1000}$
				
				$ = 0.0004 - \frac{1.0139594797900291}{1000}$
				$ = -0.00061395947979$
				
				$f(0.01) = (0.01^2 - \frac{2^{0.01}}{1000} $
				
				$= 0.0001 - \frac{1.0069555500567188}{1000}$
				$ = -0.0009069555500567$
				
				$f(0.005) = (0.005^2 - \frac{2^{0.005}}{1000} $
				
				$= 0.000025 - \frac{1.0034717485095028}{1000}$
				$ = -0.0009784717485095$
				
				$f(0.003) = (0.003^2 - \frac{2^{0.003}}{1000} $
				
				$= 0.000009 - \frac{1.0020816050796328}{1000}$
				$ = -0.0009930816050796$
				
				$f(0.001) = (0.001^2 - \frac{2^{0.001}}{1000} $
				
				$= 0.000001 - \frac{1.0006933874625806}{1000}$
				$ = -0.0009996933874626$
				
				It now looks like  $\lim \limits _{x \to 0} (x^2 - \frac{2^x}{1000}) = -0.001$
			\end{enumerate}
			
			\item 
			
			\begin{enumerate}
			
				\item Evaluate $h(x) = (\tan x - x)/x^3$ for
				$x = 1, 0.5, 0.1, 0.05, 0.01, 0.005$
				
				$h(1) = (\tan 1 - 1)/1^3 = 0.557407724655$
				$h(0.5) = 0.0057878112305$
				$h(0.1) = 0.000000334672085$
				
				\item Guess the value of $\lim \limits _{x \to 0} \frac{\tan x - x}{x^3}$
				
				\item Evaluate $h(x)$ for successively smaller values of $x$ until you finally reach
					0 values for $h(x)$. Are you still confident that your guess in part (b) is correct?
					Explain why eventually obtained 0 values.
					
				\item Graph the function $h$ in the viewing rectangle $[-1,1]$ by $[0,1]$ Then zoom in
					toward the point where the graph crosses the $y$-axis to estimate the limit of 
					$h(x)$ as $x$ approaches 0. Continue to zoom in until you observe distortions in
					the graph of $h$. Compare with the results of part (c).
			\end{enumerate}
			

			\item Use the given graph of $f(x) = \sqrt{x}$ to find a number $\delta$ such that
			
				$$\text{if } | x - 4 | < \delta\text{ then } |\sqrt{x} - 2 | < 0.4$$

		\item Use the given graph of $f(x) = x^2$ to find a number $\delta$ such that
		
			$$\text{if } |x - 1| < \delta \text{ then } |x^2 - 1| < \frac{1}{2}$$
			
				So if $y = x^2$ then $x = \sqrt{y}$. If we take $y$ to be 1.5 and 0.5 (The limit of the function as $x$ approaches
				$a$ plus the maximum distance from the limit $\epsilon$ where $a = 1$ and $\epsilon = \pm 0.5$)
				
				$$x = \sqrt{1.5} = 1.22474487139 \text{ or } x = \sqrt{0.5} = 0.70710678118$$
				
%<<<<<<< HEAD
				$$| 1.22474487139 - 1| = 0.22474487139 $$
				$$ \text{or } |0.70710678118 - 1 | = 0.29289321881$$
				
				So $\delta = 0.22474487139$.
				
		\item Use a graph to find a number $\delta$ such that
		
			$$\text{ if } | x - \frac{\pi}{4} | < \delta \text{ then } | \tan x - 1 | < 0.2$$
			
		\item Use a graph to find a number $\delta$ such that 
		
			$$\text{if } | x - 1 | < \delta \text{ then } | \frac{2x}{x^2 + 4} - 0.4 | < 0.1$$
			
		\item A machinist is required to manufacture a circular metal disk with area 1000$\text{cm}^2$.
		
			\begin{enumerate}
				\item What radius produces such a disk?
				
					$$ 1000 = \pi r^2$$
					$$r = \sqrt{\frac{1000}{\pi}}$$
					$$r = 17.8412411615$$
					
				\item If the machinist is allowed an error tolerance of $\pm 5 \text{cm}^2$ in the area of the disk,
					how close to the ideal radius in part (a) must the machinist control the radius?
					
					$$ \sqrt{\frac{1005}{\pi}} = 17.8857886495$$
					$$ \sqrt{\frac{995}{\pi}} = 17.7965821649$$
					
					So
					
					$$|17.8857886495 - 17.8412411615| = 0.044547488 $$
					$$|17.7965821649 - 17.8412411615 = 0.0446589966$$
					
					So the radius must be $17.8412411615 \pm 0.044547488$

				\item In terms of the $\epsilon$, $\delta$ definition of $\lim \limits _{x \to a} f(x) = L$, what is $x$?
					What is $f(x)$? What is $a$? What is $L$? What is the value of $\epsilon$ given? What is the
					corresponding value of $\delta$?
					
					$x$ is the radius. $f(x)$ is the area of the circle $\pi r^2$. $a$ is the value we must make the 
					radius close to, in this case 17.8412411615. $L$ is the value we are trying to make the area close
					enough to, in this case 1000. $\delta$ is the distane from $a$ we must be, in this case $\pm 0.044547488$.
					$\epsilon$ the distance from $L$ we need to get, in this case $\pm 5$.

			\end{enumerate}
			
		\item A crystal growth furnace is used in research to determine how best to manufacture crystals used in electronic
			components for the space shuttle. For proper growth of the crystal, the temperature must be controlled 
			accordingly by adjusting the input power. Suppose the relationship is given by
			
				$$T(w) = 0.1w^2 + 2.155w + 20$$
				
			where $T$ is this temperature is degrees Celsius and $w$ is the power input in watts.
			
		\begin{enumerate}
		
			\item How much power is needed to maintain the temperature at 200 $^{\circ}$C?
			
			\item If the temperature is allowed to vary from 200$^{\circ}$C by up to $\pm 1^{\circ}\text{C}$,
				what range of wattage is allowed for the input power?
				
			\item In terms of the $\epsilon$, $\delta$ definition of $\lim \limits _{x \to a} f(x) = L$, what is $x$?
				What is $f(x)$? What is $a$? What is $L$? What value of $\epsilon$ is given? What is the 
				corresponding value of $\delta$?
		\end{enumerate}
				
		\item[29--32] Prove the statement using the $\epsilon$, $\delta$ definition of a limit and illustrate with a diagram
				like Figure 15.
				
		\item $\lim \limits _{x \to 3} (1 + \frac{1}{3}x) = 2$ 
		
			$$| x - 3| < \delta$$
			$$|\frac{1}{3}x - 1 | < \epsilon$$
			
			so
			
			$$3|x - 3| < \epsilon \text{ and } |x - 3| < \frac{\epsilon}{3} \text{ therefore } \delta < \frac{\epsilon}{3}$$

				$$| 1.22474487139 - 1| = 0.22474487139 $$ 
				$$\text{ or } |0.70710678118 - 1 | = 0.29289321881$$
				
				So $\delta = 0.22474487139$.
				
				
			
		\item[33--44] Prove the statement using the $\epsilon$, $\delta$ definition of a limit.
		
		\item $\lim \limits _{x \to 1} \frac{2 + 4x}{3} = 2$
		
			First we need a definition for $\delta$
			
			$$|x - 1| < \delta$$
			$$|\frac{2 + 4x}{3} -2| < \epsilon$$
			
			$$|2 + 4x - 6| < 3\epsilon$$
			$$|4x - 4| < 3\epsilon$$
			$$4|x - 1| < 3\epsilon$$
			$$|x - 1| < \frac{3}{4}\epsilon$$
			
			$$\text{if } |x - 1| < \delta \text{ then } \frac{4 |x - 1|}{3} < \epsilon$$
			$$|\frac{4x - 4}{3} | < \epsilon$$
			$$|\frac{4x + 2}{3} - 2| < \epsilon$$
			$$\text{so } \lim \limits _{x \to 1} \frac{4x + 2}{3} = 2$$
				
		\item $\lim \limits _{x \to 6} (\frac{1}{2}x + 3) = \frac{9}{2}$
		
			$$|\frac{x}{4} + 3 - \frac{9}{2}| < \epsilon$$
			$$|\frac{x}{4} - \frac{3}{2}| < \epsilon$$
			
			$$|\frac{x}{4} - \frac{6}{4}| < \epsilon$$
			$$4|\frac{x}{4} - \frac{6}{4}| < 4\epsilon$$
			$$|x - 6| < 4\epsilon$$
			
			$$\delta = 4\epsilon$$
			
			$$\text{If } | x - 6 | < \delta \text{ then } \frac{1}{4}| x - 6| < \epsilon$$
			$$|\frac{x}{4} - \frac{6}{4} | < \epsilon$$
			$$|(\frac{x}{4} + \frac{6}{2}) - \frac{9}{2} | < \epsilon$$
			$$|( \frac{x}{4} + 3) - \frac{9}{2} | < \epsilon$$
			$$\text{ so } \lim \limits _{x \to 6} (\frac{1}{2}x + 3) = \frac{9}{2}$$
			
		\item $\lim \limits _{x \to 2} \frac{x^2 + x - 6}{x - 2} = 5$
		
			$$|\frac{x ^2 + x - 6}{x - 2} - 5| < \epsilon$$
			$$|\frac{(x+3)(x-2)}{x-2} - 5| < \epsilon$$
			$$|x + 3 - 5| < \epsilon$$
			$$|x - 2| < \epsilon$$
			
			$$\delta = \epsilon$$
			
			$$\text{If } |x - 2| < \delta \text{ then } | x - 2 | < \epsilon$$
			$$|\frac{(x+3)(x-2)}{x-2} - 5| < \epsilon$$
			
			$$\text{So } \lim \limits _{x \to 2} \frac{x^2 + x - 6}{x - 2} = 5$$
			
		\item $\lim \limits _{x \to 1.5} \frac{9 - 4x^2}{3 + 2x} = 6$
		
			$$|\frac{9 - 4x^2}{3 + 2x} - 6| < \epsilon$$
			$$|\frac{9 - 4x^2 - 6(3 + 2x)}{3 + 2x}| < \epsilon$$
			$$|\frac{9 - 4x^2 - 18 - 12x}{3 + 2x}| < \epsilon$$
			$$|\frac{(2x + 3)(2x - 3)}{2x + 3}| < \epsilon$$
			
			$$|2x - 3| < \epsilon$$
			$$|x - 1.5| < \frac{\epsilon}{2}$$
			$$ \delta = \frac{\epsilon}{2}$$
			
			$$\text{If } 0 < | x - 1.5| < \delta \text{ then } 2| x - 1.5| < \epsilon 
			\text{ so } \lim \limits _{x \to 1.5} \frac{9 - 4x^2}{3 + 2x} = 6$$
			
		\item $\lim \limits _ { x \to a} x = a$
		
			$$| x - a | < \epsilon$$
			$$| x - a | < \delta$$
			$$ \delta = \epsilon$$
			
			$$\text{If } 0 < | x - a| < \delta \text{ then } |x - a| < \epsilon \text{ so } \lim \limits _ { x \to a} x = a$$
			
			
		\item $\lim \limits _ {x \to a} c = c$
		
			$$|c - c| < \epsilon \text{ so } \epsilon > 0$$
			
			Therefore $|c - c| < \epsilon$ is satisfied independently no matter what the value of $|x - a| < \delta$.
			
		\item $\lim \limits _{x \to 0} x^2 = 0$
		
			$$|x^2 - 0| < \epsilon$$
			$$|x^2| < \epsilon$$
			$$|x| < \sqrt{\epsilon}$$
			$$|x - 0 | < \sqrt{\epsilon} \text{ so } \delta = \sqrt{\epsilon}$$
			
			$$\text{If } 0 < |x - 0| < \delta \text{ then } |x - 0| < \sqrt{\epsilon} \text{ so } 
			\lim \limits _{x \to 0} x^2 = 0$$
			
		\item $\lim \limits _{x \to 0} x^3 = 0$
		
			$$|x^3 - 0| < \epsilon$$
			$$|x^3| < \epsilon$$
			$$|x - 0| < \sqrt[3]{\epsilon}$$
			$$\delta = \sqrt[3]{\epsilon}$$
			
			$$\text{If } 0 < |x - 0|  < \delta \text{ then } |x - 0|^3 < \epsilon \text { so }
			\lim \limits _{x \to 0} x^3 = 0$$
			
		\item $\lim \limits _{x \to 0} | x | = 0$
		
			So $|| x | - 0 | < \epsilon$
			
			 and $| x - 0 | < \epsilon$
			
			Therefore 
			
				$$ \delta = \epsilon$$
				
			$$\text{If } |x - 0 | < \delta \text{ then } |x - 0| < \epsilon \text{ so } ||x| - 0| < \epsilon$$
		
			
		\item $\lim \limits _{x \to 9^{-}} \sqrt[4]{9 - x} = 0$
		
			Consider $\epsilon > 0$, arbitrary. We need to find $\delta > 0$ so that for all $9 - \delta < x < 9$
			we have $|\sqrt[4]{9 - x}- 0| < \epsilon$.
			
			If we manipulate the inequality...
			
			$$\sqrt[4]{9 - x} < \epsilon$$
			$$9 - x < \epsilon^{4}$$
			$$- x < \epsilon ^{4} - 9$$
			
			 Multiply both sides by -1.
			 
			  $$9 - \epsilon^{4} < x $$ 
			  
			  So $9 - \delta = 9 - \epsilon^{4}$ therefore $\delta = \epsilon^{4}$.
			  
			  Thus for every $9 - \delta < x < 9$ there exists 
			  
			  $$9 - \epsilon^{4} < x $$
			  $$ - \epsilon^{4} < x - 9$$
			  $$ 9 - x < \epsilon^{4}$$
			  $$\sqrt[4]{9 - x} < \epsilon$$
			  $$|\sqrt[4]{9- x} - 0 | < \epsilon$$
			  
			  And so $\lim \limits _{x \to 9^{-}} \sqrt[4]{9 - x} = 0$.
	\end{enumerate}
\end{document}