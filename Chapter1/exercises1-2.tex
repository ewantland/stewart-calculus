\documentclass{article}

\renewcommand{\thesection}{} % remove section labels

\usepackage{amsmath}

\begin{document}
	\section{1.2 Exercises}
	
	\begin{enumerate}
		\item
		\begin{enumerate}
			\item Find an equation for the family of linear functions with slope 2
				and sketch several members of the family.
				
			\item Find an equation for the family of linear functions such that
				$f(2) = 1$ and sketch several members of the family.
				
			\item Which function belongs to both families.
		\end{enumerate}
		
		\item What do all members of the family of linear functions $f(x) = c - x$ have in common?
			Sketch several members of the family.
			
		\item Find expressions for the quadratic functions whose graphs are shown.
		
			$$f(x) = 2(x - 3)^2$$
			
			For the second graph from the point $(0, 1)$ we know the \emph{y}-intercept is 1.
			
			We know that the equation for this graph must be $ax^2 + bx + 1$.
			
			For the point $(-2, 2)$.
			
			$$a(-2)^2 + b(-2) + 1 = 2$$
			$$4a - 2b = 1$$
			
			For the point $(1, -2.5)$.
			
			$$a + b + 1 = -2.5$$
			$$a + b = -3.5$$
			$$a = -b - 3.5$$
			
			Plugging this into equation 1.
			
			$$4(-b -3.5) - 2b = 1$$
			$$-4b - 14 - 2b = 1$$
			$$-6b = 15$$
			$$b = - 2.5$$
			
			Finding \emph{a}.
			
			$$a -2.5 = -3.5$$
			$$ a = - 1$$
			
			So the equation is $-x^2 - 2.5x + 1$.
			
		\item Find an expression for a cubic function \emph{f} if $f(1) = 6$ and $f(-1) = f(0) = f(2) = 0$.
		
			So the the factors must be $x(x + 1)(x-2)$.
			
			This makes $x(x^2 - x - 2) = x^3 - x^2 - 2x$
			
			With this function $f(1) = -2$ so we can multiply by a factor of -3.
			
			$-3x(x^2 - x - 2) = -3x^3 + 3x^2 + 6x$
			
		\item Some scientists believe that the average surface temperature of the world has been rising
			steadily. They have modelled the temperature by the linear function $T = 0.02t + 8.50$,
			where \emph{T} is temperature in $^{\circ}$C and t represents years since 1900.
			
			\begin{enumerate}
				\item What do the slope and the \emph{T}-intercept represent?
				
					The slope represents the change in temperature per year. The \emph{T}-intercept
					represents the temperature at the year 1900.
					
				\item Use the equation to predict the average global surface temperature in 2100.
				
					$$t = 2100 - 1900 = 200$$
					$$0.02(200) + 8.50 = 12.5$$
				
			\end{enumerate}
			
		\item If the recommended adult dosage for a drug is \emph{D} (in mg), then to determine the appropriate
			dosage \emph{c} for a child of age \emph{a}, pharmacists use the equation $c = 0.0417D(a + 1)$.
			Suppose the dosage for an adult is 200 mg.
			
			\begin{enumerate}
				\item Find the slope of the graph of \emph{c}. What does it represent?
				
				$$c = 0.0417aD + 0.0417D$$
				$$c = 8.34D + 8.34$$
				
				The slope is 8.43. This represents mg per year.
				
				\item What is the dosage for a newborn?
				
				8.43mg as this is the \emph{c}-intercept.
			\end{enumerate}
			
		\item The manager of a weekend flea market knows from past experience that if he charges \emph{x} dollars
			for a rental space at the flea market, then the number of \emph{y} spaces he can rent is given by the
			equation $y = 200 - 4x$.
			
			\begin{enumerate}
				\item Sketch a graph of this linear function. (Remember that the rental charge per space and the
					number of spaces rented can't be negative quantities.)
					
				\item What do the slope, the \emph{y}-intercept, and the \emph{x}-intercept of the graph represent?
				
			\end{enumerate}
			
		\item The relationship between the Fahrenheit (\emph{F}) and Celsius (\emph{C}) temperature scales is given
			by the linear function $F = \frac{9}{5}C + 32$.
			
			\begin{enumerate}
				\item Sketch a graph of this function.
				
				\item What is the slope of the graph and what does it represent? What is the \emph{F}-intercept
					and what does it represent?
			\end{enumerate}
			
		\item Kelly leaves
	\end{enumerate}
\end{document}