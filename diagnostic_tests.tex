\documentclass{article}
\renewcommand{\thesection}{}% Remove section references...
\renewcommand{\thesubsection}{\Alph{subsection}}%... from subsections
\usepackage{amsmath}
\begin{document}
\section{Diagnostic Tests}
\subsection{Diagnostic Test: Algebra}

\begin{enumerate}
	\item Evaluate each expression without using a calculator.
	\begin{enumerate}
		\item $(-3)^4$
		
		$ (-3)\times(-3)\times(-3)\times(-3) = 9\times9 = 81 $
		
		\item $-3^4$
		
		$ -(3\times3\times3\times3) = -81 $
		
		\item $3^{-4}$
		
		$ 3^{-4} = \frac{1}{3^4} = \frac{1}{81} $
		
		\item $\frac{5^{25}}{5^{21}}$
		
		$\frac{5^{23}}{5^{21}} 
		 = 5^{23}\times5^{21} 
		 = 5^{23-21}
		 = 5^2 
		 = 25$
		 
		 \item $(\frac{2}{3})^{-2}$
		 
		 $(\frac{2}{3})^{-2} = \frac{2^{-2}}{3^{-2}} = \frac{3^2}{2^2} = \frac{9}{4}$
		 
		 \item $16^{-3/4}$
		 
		 $16^{-3/4} = \frac{1}{16^{3/4}} = \frac{1}{(\sqrt[4]{16})^3} = \frac{1}{2^3} = \frac{1}{8}$
		 
	\end{enumerate}
	\item Simplify each expression. Write your answer without negative exponents.
	\begin{enumerate}
		\item $\sqrt{200} - \sqrt{32}$
		
		$\sqrt{200} - \sqrt{32} =  \sqrt{2^2\times5^2\times2} - \sqrt{2^2\times2^2\times2} = 10\sqrt{2} - 4\sqrt{2} = 6\sqrt{2}$
		
		\item $(3{a}^{3}b^{3})(4ab^{2})^{2}$
		
		$(3{a}^{3}b^{3})(4ab^{2})^{2} = (3{a}^{3}b^{3})(16a^2b^4) = 48a^3a^2b^3b^4 = 48a^5b^7$
		
		\item $(\frac{3x^{3/2}y^{3}}{x^{2}y^{-1/2}})^{-2}$
		
		$(\frac{3x^{3/2}y^{3}}{x^{2}y^{-1/2}})^{-2} = (\frac{x^{2}y^{-1/2}}{3x^{3/2}y^{3}})^{2} = \frac{x^4y^{-1}}{9x^3y^6} = \frac{x}{9y^7}$
	\end{enumerate}
	\item Expand and simplify.
	\begin{enumerate}
		\item $3(x + 6) + 4(2x - 5) $
		
		$ 3(x + 6) + 4(2x - 5) = 3x + 18 + 8x - 20 = 11x - 2 $
		
		\item $(x + 3)(4x - 5)$
		
		$(x + 3)(4x - 5) = 4x^2 - 5x + 12x - 15 = 4x^2 + 7x - 15$
		
		\item $(\sqrt{a} + \sqrt{b})(\sqrt{a} - \sqrt{b})$
		
		From the difference of two squares $a^2 - b^2 = (a + b)(a - b)$
		
		$$ (\sqrt{a} + \sqrt{b})(\sqrt{a} - \sqrt{b}) = a - b $$
		
		\item $(2x + 3)^2$
		
		$(2x + 3)^2 = (2x + 3)(2x + 3) = 4x^2 + 12x + 9$
		
		\item $(x + 2)^3$
		
		From the binomial theorem we know that third order binomials look like
		
		$$ a^3 + 3a^2b + 3ab^2 + b^3 $$
		
		$ x^3 + 3x^2(2) + 3x(2)^2 + 2^3 = x^3 + 6x^2 + 12x + 8 $
	\end{enumerate}
	\item Factor each expression.
	\begin{enumerate}
		\item $4x^2 - 25$
		
		From the difference of two squares $a^2 - b^2 = (a + b)(a - b)$
		
		$4x^2 - 25 = (2x + 5)(2x - 5)$
		
		\item $2x^2 + 5x - 12$
		
		$2x^2 + 5x - 12 = (2x - 3)(x + 4)$
		
		\item $x^3 - 3x^2 - 4x + 12$
		
		From the factor theorem we need to find a value for $x$ where this equation equals zero.
		
		If $x = 1$
		
		$(1)^3 - 3(1)^2 - 4(1) + 12 = 1 - 3 - 4 + 12 = 5$
		
		If $x = 2$
		
		$(2)^3 - 3(2)^2 - 4(2) + 12 = 8 - 12 - 8 + 12 = 0$
		
		We know now our first factor is $(x - 2)$
		
		$$x^3 - 3x^2 - 4x + 12 = (x - 2)(x^2 - x - 6)$$
		
		Factorising $(x^2 - x - 6)$ we get...
		
		$$(x - 2)(x^2 - x - 6) = (x-2)(x+2)(x-3)$$
		
		\item $x^4 + 27x$
		
			$$x^4 + 27x = x(x^3 + 27)$$
		
			If $x = -3$
		
			$$(-3)^3 + 27 = 0$$
		
			$$x(x^3 + 27) = x(x + 3)(x^2 - 3x + 9)$$
		
		\item $3x^{3/2} - 9x^{1/2} + 6x{-1/2}$
		
			If we factor out $3x^{-1/2}$ we get...
			
			$$3x^{-1/2}(x^2 - 3x + 2)$$
			
			Factorising...
			
			$$3x^{-1/2}(x - 2)(x - 1)$$
			
			$$\frac{3(x-2)(x-1)}{\sqrt{x}}$$
			
		\item $x^3y - 4xy$
		
			We factor out the x and y terms
			
			$xy(x^2 - 4)$
			
			And factorise the brackets according to difference of squares.
			
			$xy(x - 2)(x + 2)$
		
	\end{enumerate}
	\item Simplify the rational expression.
	\begin{enumerate}
		\item $\frac{x^2 + 3x + 2}{x^2 - x - 2}$
		
			Factorising the top and bottom.
			
			$$\frac{(x + 2)(x + 1)}{(x-2)(x+1)}$$
			
			Cancelling out the factors.
			
			$$\frac{x+2}{x-2}$$
			
		\item $\frac{2x^2 - x - 1}{x^2 - 9} \bullet \frac{x+3}{2x + 1}$
		
			Factoring the left hand side.
			
			$$\frac{(2x + 1)(x - 1)}{(x + 3)(x - 3)} \bullet \frac{x+3}{2x+1}$$
			
			Cancelling out the factors.
			
			$$\frac{x-1}{x-3}$$
			
		\item $\frac{x^2}{x^2 - 4} - \frac{x+1}{x+2}$
		
			$$\frac{x^2}{(x + 2)(x - 2)} - \frac{x+1}{x+2}$$
			$$\frac{x^2}{(x+2)(x-2)} - \frac{(x+1)(x-2)}{(x+2)(x-2)}$$
			$$\frac{x^2 - (x+1)(x-2)}{(x+2)(x-2)}$$
			$$\frac{x^2 - (x^2 - x - 2)}{(x+2)(x-2)}$$
			$$\frac{-x-2}{(x+2)(x-2)}$$
			$$-\frac{x+2}{(x+2)(x-2)}$$
			$$-\frac{1}{x-2}$$
			
		\item $\frac{\frac{y}{x} - \frac{x}{y}}{\frac{1}{y} - \frac{1}{x}}$

			$$(\frac{y}{x} - \frac{x}{y}) \div (\frac{1}{y} - \frac{1}{x})$$

			Cross multiplying the denominator and numerators...

			$$(\frac{y^2 - x^2}{xy}) \div (\frac{x-y}{xy})$$

			Flipping the fraction and changing operator

			$$(\frac{y^2 - x^2}{xy}) \times (\frac{xy}{x-y})$$

			Cancel out common factors and merge...

			$$\frac{y^2 - x^2}{x-y}$$

			Multiply by -1

			$$-\frac{x^2 - y^2}{x-y}$$

			Difference of squares

			$$-\frac{(x+y)(x-y)}{x-y}$$

			Cancelling out common factors

			$$-(x+y)$$
		
	\end{enumerate}
	\item Rationalize the expression and simplify.
	\begin{enumerate}
		\item $\frac{\sqrt{10}}{\sqrt{5} - 2}$

			We must multiply both top and bottom to create a difference of squares.

			$$\frac{\sqrt{10}}{\sqrt{5} - 2} \times \frac{\sqrt{5}+2}{\sqrt{5}+2}$$

			$$\frac{\sqrt{50} + 2\sqrt{5}}{5 - 4}$$

			$$5\sqrt{2} + 2\sqrt{5}$$

		\item $\frac{\sqrt{4 + h} - 2}{h}$

			Multiplying top and bottom

			$$\frac{\sqrt{4 + h} - 2}{h} \times \frac{\sqrt{4+h} + 2}{\sqrt{4+h} + 2}$$

			$$\frac{4 + h - 4}{h\sqrt{4 + h} + 2h}$$

			$$\frac{1}{\sqrt{4 + h} + 2}$$
	\end{enumerate}
	\item Rewrite by completing the square

	\begin{enumerate}
		\item $x^2 + x + 1$

			$$x^2 + x = -1$$
			$$x^2 + x + \frac{1}{4} = -\frac{3}{4}$$
			$$(x + \frac{1}{2})^2 = -\frac{3}{4}$$
			$$(x + \frac{1}{2})^2 + \frac{3}{4}$$

		\item $2x^2 - 12x + 11$

			$$2x^2 -12x = -11$$
			$$2(x^2 - 6x) = -11$$
			$$2(x^2 -6x + 9) = 7$$
			$$2(x - 3)^2 = 7$$
			$$2(x-3)^2 - 7$$
	\end{enumerate}
	\item Solve the equation. (Find only the real solutions.)
	\begin{enumerate}
		\item $x + 5 = 14 - \frac{1}{2}x$

			$$\frac{3}{2}x + 5 = 14$$

			$$\frac{3}{2}x = 9$$

			$$3x = 18$$

			$$x = 6$$

		\item $\frac{2x}{x + 1} = \frac{2x-1}{x}$
			$$2x^2 = (2x + 1)(x+1)$$
			$$2x^2 = 2x^2 + 3x + 1$$
			$$3x + 1 = 0$$
			$$3x = -1$$
			$$x = -\frac{1}{3}$$

		\item $ x^2 - x - 12$
			$$x^2 - x - 12 = (x-4)(x+3)$$

			$x = 4$ or $x = -3$

		\item $2x^2 + 4x + 1 = 0$

			Completing the square.

			$$2x^2 + 4x = -1$$
			$$2(x^2 + 2x) = -1$$
			$$x^2 + 2x = - \frac{1}{2}$$
			$$x^2 + 2x + 1 = \frac{1}{2}$$
			$$(x+1)^2 = \frac{1}{2}$$
			$$x + 1 = \pm \sqrt{\frac{1}{2}}$$
			$$ x = -1 \pm \sqrt{\frac{1}{2}}$$

		\item $x^4 - 3x^2 + 2 = 0$

			$$(x^2)^2 - 3(x^2) + 2 = 0$$
			$$(x^2-2)(x^2-1)$$
			$$(x^2-2)(x+1)(x-1)$$
			$x = \pm \sqrt{2}$ or $x = \pm 1$

		\item $3|x - 4| = 10$

			$$|x - 4| = \frac{10}{3}$$
		
		For the positive case...
			$$x = 4 + \frac{10}{3}$$

		For the negative case...
			$$x - 4 = -\frac{10}{3}$$

			$x = 4 \pm \frac{10}{3}$

		\item $2x(4-x)^{-1/2} - 3\sqrt{4-x} = 0$

			$$2x(4-x)^{-1/2} = 3\sqrt{4-x}$$
			$$2x = 3(4-x)$$
			$$2x = 12 - 3x$$
			$$5x = 12$$
			$$x = \frac{12}{5}$$
		
	\end{enumerate}
	\item Solve each inequality. Write your answer using interval notation.
	\begin{enumerate}

		\item $-4 < 5 - 3x \leq 17$

		Multiply all by negative 1.

		$$4 > 3x - 5 \geq -17$$
		$$9 > 3x \geq -12$$
		$$3 > x \geq -4$$
		$$ = [-4, 3)$$

		\item $x^2 < 2x +8$

			$$ x^2 -2x - 8 < 0$$
			$$ (x-4)(x+2) < 0$$

			$x-4$ is positive when $x = 4$. $x + 2$ is positive when $x = -2$.

			$$(-2, 4)$$

		\item $x(x - 1)(x+2) > 0$

			$(-2, 0)\cup (1, \infty)$

		\item $|x - 4| < 3$

			$x - 4 < 3$ so $x <7$
			$x - 4 > -3$ so $x > 1$
			$$(1,7)$$

		\item $\frac{2x-3}{x+1} \leq 1$

			$$2x-3 \leq x+1$$
			$$x - 3 \leq 1$$
			$$x \leq 4$$
			$$(-\infty, 4]$$

	\end{enumerate}

	\item State whether each equation is true or false

	\begin{enumerate}

		\item $(p + q)^2 = p^2 + q^2$
		
			False. $(p + q)^2 = p^2 + 2pq + q^2$

		\item $\sqrt{ab} = \sqrt{a}\sqrt{b}$

			This is true.
			$\sqrt{2}\sqrt{2} = (\sqrt{2})^2 = 2$
			$\sqrt{2\times2} = \sqrt{4} = 2$

		\item $\sqrt{a^2 + b^2} = a +  b$

			If this is true we should be able to algebraically flip the expressions round.

			$$a^2 + b^2 = (a + b)^2$$
			$$a^2 + b^2 = a^2 + 2ab + b^2$$

			This is obviously false.

		\item $\frac{1 + TC}{C} = 1 + T$

			$$\frac{1 + TC}{C} = \frac{1}{C} + T$$

			So this is false.

		\item $\frac{1}{x - y} = \frac{1}{x} - \frac{1}{y}$

			False

		\item $\frac{1/x}{a/x - b/x} = \frac{1}{a-b}$

			$$\frac{1}{x} \div (\frac{a - b}{x})$$
			$$\frac{1}{x} \times (\frac{x}{a-b}$$
			$$\frac{1}{a-b}$$

			True.
	\end{enumerate}

\end{enumerate}

\newpage

\section{Diagnostic Test: Analytic Geometry}

\begin{enumerate}

\item Find an equation for the line that passes through the point (2, -5) and

	\begin{enumerate}
		\item has slope -3

		The point slope form of a line is

		$$(y - y_1) = m(x - x_1)$$

		Adding in our values and rearranging to slope intercept we get...

		$$(y - (-5)) = -3(x - 2)$$
		$$y = -3x + 1$$

		\item as parallel to the $x$-axis.

			When parallel to the x axis gradient is zero so....

			$$(y - (-5)) = 0$$
			$$y = -5$$

		\item is parallel to the $y$-axis

			When parallel to the $y$-axis $x = 2$

		\item is parallel to the line $2x - 4y = 3$

			Lines that are parallel have the same slope $m$ so we must find the slope of our line by rearranging to
			slope-intercept form.

			$$2x - 4y = 3$$
			$$-4y = -2x + 3$$
			$$y = \frac{1}{2}x - \frac{3}{4}$$

			$m= \frac{1}{2}$ so plugging this into a point-slope form equations gives...

			$$(y - (-5)) = \frac{1}{2}(x - 2)$$

			Rearranging to slope intercept...

			$$y = \frac{1}{2}x - 6$$

			
	\end{enumerate}

          \item Find an equation for the circle that has centre (-1, 4) and passes through the point (3, -2).

	Calculating $r^2$

	$$r^2 = (-1-3)^2 + (4 - (-2))^2 = (-4)^2 + 6^2 = 16 + 36 = 52$$

	The standard form equation for a circle is

	$$(x-a)^2 + (y-b)^2 = r^2$$

	Plugging our values in we get...

	$$(x + 1)^2 + (y  - 4)^2 = 52$$

	\item Find the center and radius of the circle with equation $x^2 + y^2 - 6x + 10y + 9 = 0$.

		This is the general of a circle equation. We want to express this in standard form.

		$$x^2 + y^2 - 6x + 10y = -9$$
		Complete the square for $x$...

		$$(x^2 - 6x + 9) + y^2 + 10y = 0$$
		$$(x - 3)^2 + y^2 + 10y = 1$$

		Complete the square for y...

		$$(x-3)^2 + (y^2 + 10y + 25) = 25$$
		$$(x-3)^2 + (y + 5)^2 = 25$$

		So $a = 3$, $b = -5$ and $r= 5$ 

	\item Let $A(-7, 4)$ and $B(5, -12)$ be points in the plane.

	\begin{enumerate}

		\item Find the slope of the line that contains $A$ and $B$.

		Using the formula $m=\frac{y_1 - y_2}{x_1 - x_2}$...

		$$\frac{4 -(-12)}{-7 - 5} = \frac{16}{-12} = -1\frac{1}{3}$$

		\item Find an equation of the line that passes through $A$ and $B$. What are the intercepts?

			We can take a point-slope equation with our value for $m$ and rearrange to get slope-intercept.

			$$(y - 4) = -\frac{4}{3}(x - (-7))$$
			$$y = \frac{4}{3}x - 5\frac{1}{3}$$

			$y$-intercept is at $x=-5\frac{1}{3}$. $x$-intercept at $x = -4$.

	\item Find the midpoint of segment $AB$.

			The midpoint can't be found by taking the mean of the $x$ and the mean of the $y$ values.

			Calculating the $x$ point...

			$$\frac{-7 + 5}{2} = \frac{-2}{2} = -1$$

			Calculate the $y$ point...

			$$\frac{4 + (-12)}{2} = \frac{-8}{2} = -4$$

			The midpoint of $AB$ is at $(-1, -4)$.

	\item Find the length of segment $AB$.

		Using pythagoras theorem...

		$$\sqrt{(-7 - 5)^2 + (4 - (-12))^2} = \sqrt{(-12)^2 + 16^2}$$
		$$\sqrt{144 + 256} = \sqrt{400} = 20$$

	\item Find an equation of the perpendicular bisector of $AB$.

		The perpendicular bisector willl have slope $-\frac{1}{m}$ and pass through midpoint.

		Gradient of bisector = $-\frac{1}{-4/3} = \frac{3}{4}$

		Using the favorite point slope equation format with midpoint...

		$$(y - (-4)) = \frac{3}{4}(x - (-1))$$
		$$y + 4 = \frac{3}{4}x + \frac{3}{4}$$
		$$y = \frac{3}{4}x - \frac{13}{4}$$
		Expressing in standard form...
		$$4y - 3x + 13 = 0$$

	\item Find an equation of the circle for which $AB$ is a diameter.

		So the midpoint must be the centre of the circle and the radius is half the length of the line.

		$$(x + 1)^2 + (y + 4)^2 = 400$$

	\end{enumerate}

	\item Sketch the region defined in the $xy$ plane defined by the equation or inequalities.

		\begin{enumerate}

			\item $1 \leq y \leq 3$

			\item $|x| < 4$ and $|y| < 2$

			\item $y < 1 - \frac{1}{2}x$

			\item $y \geq x^2 - 1$

			\item $x^2 + y^2 < 4$

			\item $9x^2 + 16y^2 = 144$

		\end{enumerate}

\end{enumerate}

\newpage

\section{Diagnostic Test: Functions}

\begin{enumerate}
	\item The graph of a function $f$ is given at the left.

	\begin{enumerate}
		\item State the value of $f(-1)$.

			2

		\item Estimate the value of $f(2)$.

			2.8

		\item For what values of $x$ is $f(x) = 2$?

			-3 and 2.

		\item Estimate the values of $x$ such that $f(x) = 0$.

			-2.5 and 0.3.

		\item State the domain and range of $x$.

			Domain = [-3, 3]
			Range = [-2, 3]
		
	\end{enumerate}

	\item If $f(x) = x^{3}$, evaluate the difference quotient $\frac{f(2 + h) - f(2)}{h}$ and simplify your answer.

		$$\frac{f(2 + h) - f(2)}{h}$$
		$$\frac{(2 + h)^3 - 2^3}{h}$$
		$$(2 + h)^3 = 2^3 + 3(2)^2h + 3(2)h^2 +  h^3 = 8 + 12h + 6h^2 + h^3$$
		$$\frac{ 8 + 12h + 6h^2 + h^3 - 8}{h}$$
		$$\frac{12h + 6h^2 + h^3}{h} = h^2 + 6h + 12$$

	\item Find the domain of the function.

	\begin{enumerate}
		\item $f(x) = \frac{2x + 1}{x^2 + x - 2}$

			Factorising the denominator.

			$$f(x) = \frac{2x + 1}{(x+2)(x-1)}$$
			
			Denominator can't be zero so domain = $(-\infty, -2) \cup (-2, 1) \cup (1,\infty)$

		\item $g(x) = \frac{\sqrt[3]{x}}{x^2 + 1}$

			$$(-\infty, \infty)$$

		\item $h(x) = \sqrt{4 - x} + \sqrt{x^2 - 1}$

		$4 - x \geq 0$ and $x^2 - 1 \geq 0$

		So $(-\infty, -1] \cup [1, 4]$
	\end{enumerate}

	\item How are graphs of the functions obtained from the graph of $f$?
	\begin{enumerate}
		\item $y = -f(x)$

			Graph is reflected about the x axis.

		\item $y = 2f(x) - 1$

			Graph is scaled in the y direction by a factor of 2 and translated by -1 in the y direction.

		\item $y = f(x - 3) + 2$

			Graph is translated up 2 in the $y$ direction and right 3 in the $x$ direction.
	\end{enumerate}

	\item Without using a calculator, make a rough sketch of the graph
	\begin{enumerate}
		\item $y = x^3$

		\item $y = (x + 1)^{3}$

		\item $y = (x - 2)^{3} + 3$

		\item $ y = 4 - x^{2} $

		\item $y = \sqrt{x}$

		\item $y = 2\sqrt{x}$

		\item $ y = -2^{x}$

		\item $ y = 1 + x^{-1} $
	\end{enumerate}

	\item Let 
 \begin{displaymath}
   f(x) = \left\{
     \begin{array}{lr}
       1 - x^{2} & \text{if } x \leq 0\\
       2x + 1 &  \text{if } x > 0 
     \end{array}
   \right.
\end{displaymath} 

	\begin{enumerate}
		\item Evaluate $f(-2)$ and $f(1)$.

		In $f(-2) x = -2$ so we use the second case $2x + 1$. Plugging in -2 for $x$ results in -3.

		For $f(1) x = 1$ so we use the first case $ 1 - x^{2} $. Plugging in 1 for $x$ results in 0.

		\item Sketch the graph of $f$.
	\end{enumerate}

	\item If $ f(x) = x^2 + 2x - 1$ and $g(x) = 2x - 3$, find each of the following functions.

	\begin{enumerate}
		\item $f \circ g$

			$$ (2x - 3)^ 2 + 2(2x-3) - 1 $$
			$$ 4x^2 - 12x + 9 + 4x - 6 - 1$$
			$$ 4x^2 - 8x + 2 $$

		\item $g \circ f$

			$$ 2(x^2 + 2x - 1) - 3$$
			$$ 2x^2 + 4x - 2 - 3$$
			$$ 2x^2 +4x -5 $$

		\item $g \circ g \circ g$

			$$ 2(2x - 3) - 3$$
			$$ 4x - 6 - 3 $$
			$$ 4x - 9 $$
			$$ 2(4x - 9) - 3 $$
			$$ 8x - 18 - 3 $$
			$$ 8x - 21 $$
	\end{enumerate}
\end{enumerate}

\newpage

\section{Diagnostic Test: Trigonometry}

\begin{enumerate}
	\item Convert from degrees to radians.
	\begin{enumerate}
		\item $300^{\circ}$

			$\frac{300\pi}{180} = 5.2360$ radians
		\item $ - 28^{\circ}$

			$\frac{-28\pi}{180} = -0.4887$ radians
	\end{enumerate}
	\item Convert from radians to degrees.

	\begin{enumerate}
		\item $5\pi / 6$

			$\frac{5 \pi \times 180}{6\pi} = 150$

		\item 2

			$\frac{2 \times 180}{\pi} = 114.59...$
	\end{enumerate}

	\item Find the length of an arc of a circle with radius 12 cm if the arc subtends an angle of $30^{\circ}$.

		$a = r\theta$
		
		$\theta = \frac{30 \times \pi}{180} = \frac{\pi}{6}$

		$a = 12\frac{\pi}{6} = 2\pi$

	\item Find the exact values.
	\begin{enumerate}
		\item $\tan(\pi/3)$
			The angle is $\pi/3$ so we know the opp = $\sqrt{3}$, adj = 1 and hyp = 2.
			
			$\tan(\pi/3) = \sqrt{3}$

		\item $\sin(7\pi/6)$

			The angle is $7\pi/6$ so we know that opp = $-\sqrt{3}$, adj = --1 and hyp = 2.

			$\sin(7\pi/6) = -\frac{\sqrt{3}}{2}$

		\item $\sec(5\pi/3)$

			The angle is $5\pi/3$ so we know that opp = $-\sqrt{3}$ , adj = 1 and hyp = 2.

			$\sec(5\pi/3) = 2$
	\end{enumerate}

		\item Express the lengths $a$ and $b$ in the figure in terms of $\theta$.

			$$a = 24 \sin\theta$$
			$$b = 24 \cos\theta$$

		\item If $\sin x = \frac{1}{3}$ and $\sec y = \frac{5}{4}$, where $x$ and $y$ lie between 0 and
			$\pi/2$ , evaluate $\sin(x + y)$.

			Firstly can convert $\sec y = \frac{5}{4}$ into $\cos y = \frac{4}{5}$

			Working backwords we can use a trigonometric identity...

			$$ \sin (x + y) = \sin x\cos y + \sin y\cos x $$

			First the first term we can substitute in our values...

			$$ \frac{4}{15} + \sin y\cos{x} $$

			Now we need to find the value of $\sin y$ . From $\cos y $ we can imagine a triangle with
			hypotenuse 5 and adjacent 4, so the opposite must be $\sqrt{5^2 - 4^2} = 3$. Therefore
			$\sin y = \frac{3}{5}$ .

			Repeating for $\cos x$. From $\sin x$ we can imagine a triangle with hypotenuse 3 and
			opposite 1, so the adjacent must be $\sqrt{3^2 - 1^2} = 2\sqrt{2}$. Therefore
			$\cos x = \frac{2\sqrt{2}}{3}$

			$$ \frac{4}{15} + \frac{3}{5} \times \frac{2\sqrt{2}}{3}$$
			$$\frac{4}{15} + \frac{6\sqrt{2}}{15} = \frac{1}{15}(4 + 6\sqrt{2})$$

		\item Prove these identities.

		\begin{enumerate}
			\item $\tan \theta \sin \theta + \cos \theta = \sec \theta$

				$$ \frac{\sin \theta}{\cos \theta} \sin \theta + \cos \theta $$
				$$ \frac{\sin^2 \theta}{\cos \theta} + \cos \theta $$
				$$ \frac{1 - \cos^2 \theta}{\cos \theta} + \cos \theta $$
				$$ \frac{1}{\cos \theta} - \cos \theta + \cos \theta $$
				$$ \frac{1}{\cos \theta} $$

			\item $\frac{ 2 \tan x}{1 + tan^2 x} = \sin 2x$
				$$ \frac{2\tan x}{\sec^2 x} $$
				$$ 2 \tan x \cos^2 x $$
				$$ 2 \frac{\sin x}{\cos x}\cos^2 x $$
				$$ 2 \sin x \cos x = \sin 2x $$
		\end{enumerate}

		\item Find all values of \emph{x} such that $ \sin 2x = \sin x $ and $0 \leq x \leq 2 \pi $

			$$ \sin2x = \sin x $$
			$$ 2\sin x\cos x = \sin x $$
			$$ 2 \cos x = 1 $$
			$$ \cos x = \frac{1}{2} $$
			$$ x = \frac{\pi}{3} $$

			So the values of \emph{x} that satisfy this are 0, $\frac{\pi}{3}$, $\pi$, $\frac{5\pi}{3}$, $2\pi$.
\end{enumerate}

\end{document}